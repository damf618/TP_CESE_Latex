%%%%%%%%%%%%%%%%%%%%%%%%%%%%%%%%%%%%%%%%%
%
% Template license:
% CC BY-NC-SA 3.0 (http://creativecommons.org/licenses/by-nc-sa/3.0/)
%
%%%%%%%%%%%%%%%%%%%%%%%%%%%%%%%%%%%%%%%%%

%----------------------------------------------------------------------------------------
%	PACKAGES AND OTHER DOCUMENT CONFIGURATIONS
%----------------------------------7------------------------------------------------------

\documentclass[
11pt, % The default document font size, options: 10pt, 11pt, 12pt
%oneside, % Two side (alternating margins) for binding by default, uncomment to switch to one side
%chapterinoneline,% Have the chapter title next to the number in one single line
spanish,
singlespacing, % Single line spacing, alternatives: onehalfspacing or doublespacing
%draft, % Uncomment to enable draft mode (no pictures, no links, overfull hboxes indicated)
%nolistspacing, % If the document is onehalfspacing or doublespacing, uncomment this to set spacing in lists to single
%liststotoc, % Uncomment to add the list of figures/tables/etc to the table of contents
%toctotoc, % Uncomment to add the main table of contents to the table of contents
parskip, % Uncomment to add space between paragraphs
%codirector, % Uncomment to add a codirector to the title page
headsepline, % Uncomment to get a line under the header
]{MastersDoctoralThesis} % The class file specifying the document structure
\usepackage{multirow}
\usepackage{verbatim} % comentarios

%----------------------------------------------------------------------------------------
%	INFORMACIÓN DE LA MEMORIA
%----------------------------------------------------------------------------------------

\thesistitle{Sistema de notificación de alarma de incendio} % El títulos de la memoria, se usa en la carátula y se puede usar el cualquier lugar del documento con el comando \ttitle

% Nombre del posgrado, se usa en la carátula y se puede usar el cualquier lugar del documento con el comando \degreename
\posgrado{Carrera de Especialización en Sistemas Embebidos} 
%\posgrado{Carrera de Especialización en Internet de las Cosas} 
%\posgrado{Carrera de Especialización en Intelegencia Artificial}
%\posgrado{Maestría en Sistemas Embebidos} 
%\posgrado{Maestría en Internet de las cosas}

\author{Ing. Daniel Marquez} % Tu nombre, se usa en la carátula y se puede usar el cualquier lugar del documento con el comando \authorname

\director{Mg. Ing. Wilmer Sanz (UC)} % El nombre del director, se usa en la carátula y se puede usar el cualquier lugar del documento con el comando \dirname
%\codirector{Nombre del Codirector (pertenencia)} % El nombre del codirector si lo hubiera, se usa en la carátula y se puede usar el cualquier lugar del documento con el comando \codirname.  Para activar este campo se debe descomentar la opción "codirector" en el comando \documentclass, línea 23.

\juradoUNO{Dr. Ing. Adrián Stacul (UTN, CITEDEF)} % Nombre y pertenencia del un jurado se usa en la carátula y se puede usar el cualquier lugar del documento con el comando \jur1name
\juradoDOS{Esp. Ing. Diego Fernández (FIUBA, DEBMEDIA)} % Nombre y pertenencia del un jurado se usa en la carátula y se puede usar el cualquier lugar del documento con el comando \jur2name
\juradoTRES{Esp. Ing. Santiago Salamandri (FIUBA)} % Nombre y pertenencia del un jurado se usa en la carátula y se puede usar el cualquier lugar del documento con el comando \jur3name

\fechaINICIO{marzo de 2020}
\fechaFINAL{diciembre de 2020}


\keywords{Sistemas Embebidos, FIUBA} % Keywords for your thesis, print it elsewhere with \keywordnames


\begin{document}


\frontmatter % Use roman page numbering style (i, ii, iii, iv...) for the pre-content pages

\pagestyle{plain} % Default to the plain heading style until the thesis style is called for the body content


%----------------------------------------------------------------------------------------
%	RESUMEN - ABSTRACT 
%----------------------------------------------------------------------------------------

\begin{abstract}
\addchaptertocentry{\abstractname} % Add the abstract to the table of contents
%
%The Thesis Abstract is written here (and usually kept to just this page). The page is kept centered vertically so can expand into the blank space above the title too\ldots
\centering

La presente memoria describe la implementación de un sistema de monitoreo de alarmas de incendio, con el objetivo de satisfacer la necesidad detectada por la firma Isolse SRL: brindar a sus clientes la posibilidad de conocer de forma remota el estado del sistema de detección de incendio.

El sistema se elabora utilizando conocimientos de programación de microcontroladores e ingeniería de software, protocolos de comunicación, diseño de circuitos impresos, manejo de bases de datos y micro-servidor web.

\end{abstract}

%----------------------------------------------------------------------------------------
%	CONTENIDO DE LA MEMORIA  - AGRADECIMIENTOS
%----------------------------------------------------------------------------------------

\begin{acknowledgements}
%\addchaptertocentry{\acknowledgementname} % Descomentando esta línea se puede agregar los agradecimientos al índice
\vspace{1.5cm}

Este trabajo es el resultado del esfuerzo de muchos. Aprovecho esta sección para agradecerles...

A Dios por la vida y la salud.

Mi familia por todo su apoyo y ánimo, el saber que están ahí para mí, me impulsa a mejorar y seguir creciendo cada día.   

Bettys Farias y Reinaldo Marquez, excelentes profesionales, amigos y mis queridos padres.

Andrea Fuenmayor, mi prometida con quien comparto este logro. Mi mejor amiga quien sin entender nada buscó todas las formas posibles para ayudarme.    

Marcos Ulloa, por su creatividad y compromiso. 

Ana Elisa, Marcela, Alicia, Hortencia, los Marcos y mis primos por siempre estar conmigo. No existe mayor inspiración que soñar con volver a verlos. 

A mi familia Argentina. Evelia, Julio, Mirta, Mariana, Rubén, Isabel quienes nos cuidaron y nos ayudaron a dar nuestros primeros pasos.  

Al equipo de la firma ISOLSE SRL, por todo su apoyo, confianza y la oportunidad de poder desarrollar el proyecto.

A todo el personal del Laboratorio de Sistema Embebidos de la FIUBA, por su enorme dedicación y excelencia. 

\end{acknowledgements}

%----------------------------------------------------------------------------------------
%	LISTA DE CONTENIDOS/FIGURAS/TABLAS
%----------------------------------------------------------------------------------------
\renewcommand{\listtablename}{Índice de Tablas} %TODO eliminar esta línea

\tableofcontents % Prints the main table of contents
%
\listoffigures % Prints the list of figures
%
\listoftables % Prints the list of tables


%----------------------------------------------------------------------------------------
%	CONTENIDO DE LA MEMORIA  - DEDICATORIA
%----------------------------------------------------------------------------------------

\dedicatory{
Este proyecto con todo el tiempo, esfuerzo y cariño que conllevó se lo dedico a mis hermanos Reinaldito, Alexander, Miguel, Felipe, Alejandro, Stephanie, Anabelle, Raúl, Víctor, José, Óscar y Oriana.

Que este trabajo los enorgullezca y los inspire a alcanzar sus metas, a soñar alto y a celebrar juntos todos nuestros logros. 


}  % escribir acá si se desea una dedicatoria

%----------------------------------------------------------------------------------------
%	CONTENIDO DE LA MEMORIA  - CAPÍTULOS
%----------------------------------------------------------------------------------------

\mainmatter % Begin numeric (1,2,3...) page numbering

\pagestyle{thesis} % Return the page headers back to the "thesis" style

%\renewcommand{\tablename}{Tabla} %TODO eliminar esta línea

% Incluir los capítulos como archivos separados desde la carpeta Chapters
% Descomentar las líneas a medida que se escriben los capítulos

% Chapter 1

\chapter{Introducción general} % Main chapter title

\label{Chapter1} % For referencing the chapter elsewhere, use \ref{Chapter1} 
\label{IntroGeneral}

%----------------------------------------------------------------------------------------

% Define some commands to keep the formatting separated from the content 
\newcommand{\keyword}[1]{\textbf{#1}}
\newcommand{\tabhead}[1]{\textbf{#1}}
\newcommand{\code}[1]{\texttt{#1}}
\newcommand{\file}[1]{\texttt{\bfseries#1}}
\newcommand{\option}[1]{\texttt{\itshape#1}}
\newcommand{\grados}{$^{\circ}$}

%----------------------------------------------------------------------------------------

En el presente capítulo se plantea el propósito de esta investigación como instrumento para dar respuesta a la necesidad latente de contar con un sistema de monitoreo remoto para la detección de incendio. Además se describe el objetivo general y los objetivos específicos que se desarrollaron; así como los alcances y las limitaciones presentadas durante el progreso de la misma.

%----------------------------------------------------------------------------------------

\section{Sistemas de alarmas de detección incendio}

Un elemento de gran importancia para el éxito ante situaciones de emergencia, es contar con una planificación que especifique la serie de acciones a realizar. Estos protocolos usualmente se apoyan en sistemas automatizados que brindan dos ventajas significativas: vigilancia constante de amenazas o riesgos, lo que genera un tiempo de reacción menor  y un sistema de notificación que una vez confirmada la condición de incendio notifica de manera eficiente al personal especializado y ocupantes de acuerdo a los protocolos establecidos.
El objetivo de los sistemas de detección de incendio, es reducir el impacto y las pérdidas que podrían ocasionar un incendio en el edificio o instalación a proteger. Es importante detectar una condición de incendio en el menor tiempo posible, de la manera más eficiente y segura, ya que la rápida propagación del fuego puede reducir la ventana de tiempo seguro para realizar la evacuación de las personas y disminuye el riesgo de pérdida de vidas humanas, adicionalmente estos otros beneficios que implican su instalación se pueden mencionar: proteger de bienes y servicios de la propiedad, preservar la integridad de las instalaciones, mantener la continuidad operativa del negocio, facilitar las tareas de los servicios de emergencia, además de cumplir con requerimientos específicos y regulaciones locales.

\begin{figure}[h]
	\centering
	\includegraphics[scale=.4]{./Figures/Capitulo1/FIG_A1.png}
	\caption{Ejemplo de sistema de detección de incendio complejo.}
	\label{fig:figura_a1}
\end{figure}

\section{Sistema de notificación de alarma de incendio tradicional}

Los sistemas de alarma contra incendio se han diseñado tradicionalmente para cumplir el objetivo de detectar eventos de incendio en áreas estructurales de inmuebles que no son necesariamente monitoreados de forma constante por un personal de mantenimiento, seguridad o por los propios usuarios, lo que puede afectar el tiempo de reacción, adicionalmente se requiere que el personal asignado se encuentre correctamente instruido para la identificación de los eventos.
Estos sistemas tradicionales de detección de incendios están conformados por:
Panel de control: dispositivo electrónico principal que recibe las conexiones de los diferentes artefactos que conforman al sistema, contiene toda la lógica que se debe ejecutar ante el disparo de eventos y funciona como interfaz entre el usuario y el sistema de detección de incendio.
Detectores: existen diferentes tipos de detectores cada uno con una tecnología particular para la detección de un riesgo específico, puede ser humo, temperatura, gases  o incluso elementos de activación manual como los pulsadores de alarma de incendio.
Sirenas: el aviso ante una emergencia, suele realizarse mediante el accionamiento de dispositivos de notificación sonora como sirenas o parlantes, y visual con destellos de luz estroboscópica.
Elementos de protección activa: en algunos casos se puede incluir un sistema de protección más completo con subsistemas de supresión de incendio, sistemas de comunicación de emergencia, puertas cortafuego, entre otros.

\begin{figure}[h]
	\centering
	\includegraphics[scale=.4]{./Figures/Capitulo1/FIG_B1.png}
	\caption{Ejemplo de sistema de detección de incendio complejo.}
	\label{fig:figura_a2}
\end{figure}

\section{Sistema de notificación de alarma de incendio propuesta}

El sistema propuesto es un equipo de monitoreo remoto de alarma de detección de incendio que indique a diferentes plataformas el estado de la instalación. El diseño evita que la comunicación de eventos entre usuarios se realice basados en la interpretación del personal que vigila la central, el sistema brinda a cada usuario  una interfaz para observar el estado de la instalación. De esta forma asegurarnos la veracidad de la información y somos capaces de notificar a todo el personal capacitado de la presencia de un evento y que no necesariamente tenga acceso al panel de control del sistema en ese instante.

\begin{figure}[h]
	\centering
	\includegraphics[scale=.4]{./Figures/Capitulo1/FIG_C1.png}
	\caption{Ejemplo de sistema de detección de incendio complejo.}
	\label{fig:figura_a3}
\end{figure}

\section{Motivación}

La firma ISOLSE SRL se dedica a la venta e instalación de sistemas de seguridad electrónica, con particular experiencia en el área de detección y supresión de incendios, cuenta con con una amplia gama de clientes repartidos en todo el territorio nacional lo cual hace necesario mantener una movilización constante de su personal técnico. 

Este proyecto surge de la necesidad detectada por la empresa ISOLSE SRL, de brindar a sus clientes y al personal técnico de ISOLSE SRL el estado del sistema de detección de incendios de las infraestructuras contratadas.

Es así como surge el planteamiento del diseño de un sistema de monitoreo de alarma de forma remota, que funcione en conjunto con las centrales de alarma de incendio, para la  inclusión de notificación de los usuarios a través de plataformas alternativas.

Este sistema brinda beneficios tanto al usuario al brindar un servicio de monitoreo constante de su instalación a través de plataformas web.  Mientras que a la empresa ISOLSE SRL el sistema diseñado le permitirá, planificar sus inspecciones de mantenimiento, o incluso coordinar visitas que antes podían ser emergencias producto de falsas alarmas, mejorando la atención al cliente y generando nuevas rutinas de prevención y mantenimiento del sistema.


\section{Estado del Arte}

A continuación se describen algunas de las opciones disponibles, estas opciones varían con respecto a las prestaciones que brindan a sus usuarios, pero todas enfocadas en complementar la notificación de eventos. 

\subsection{SafeLink}

La empresa Johnson Control desarrolló para su línea de centrales de alarmas de incendio de la marca Simplex, una placa opcional que se vincula a través del  protocolo de comunicación interno de la central y genera una interfaz web con el detalle de todos los eventos que ocurren en la central, al brindar conexión a internet el dispositivo, el sistema tiene la capacidad de automatizar el envío de emails a diferentes usuarios ante eventos específicos.

\begin{figure}[h]
	\centering
	\includegraphics[scale=.4]{./Figures/Capitulo1/FIG_D1.jpg}
	\caption{Ejemplo de sistema de detección de incendio complejo.}
	\label{fig:figura_d1}
\end{figure}

\subsection{Haltel HT-7001}

Un dispositivo de uso universal que permite al usuario recibir la alerta del disparo de alarma. El sistema brinda la posibilidad de que el usuario tenga sus alarmas online independientemente del lugar geográfico donde se encuentre y además brinda la oportunidad de cancelar el evento si la situación así lo requiere. 

\begin{figure}[h]
	\centering
	\includegraphics[scale=.4]{./Figures/Capitulo1/FIG_E1.jpeg}
	\caption{Ejemplo de sistema de detección de incendio complejo.}
	\label{fig:figura_e1}
\end{figure}	
	
\subsection{UNO-2483G}

Corresponde a la propuesta de la empresa SPHINX por un sistema de detección de incendio completo, en el cual se sustituye la central de incendio por un computador de automatización embebido, la conexión con dispositivos de detección y notificación se realiza mediante dispositivos de conversión analogico digital. El sistema cuenta con una plataforma de acceso web, a través de la cual notifica cualquier evento presente.  
	
	\begin{figure}[h]
	\centering
	\includegraphics[scale=.4]{./Figures/Capitulo1/FIG_F1.png}
	\caption{Ejemplo de sistema de detección de incendio complejo.}
	\label{fig:figura_f1}
\end{figure}

\subsection{Discador telefónico}

Son dispositivos utilizados para realizar llamadas a una lista de contactos configurable con mensajes pregrabados ante diferentes eventos de alarma, intrusión, emergencia médica, entre otros. 

\begin{figure}[h]
	\centering
	\includegraphics[scale=.4]{./Figures/Capitulo1/FIG_G1.jpeg}
	\caption{Ejemplo de sistema de detección de incendio complejo.}
	\label{fig:figura_g1}
\end{figure}

\subsection{Comparativa de características}

Los sistemas descritos anteriormente se pueden analizar a través de la tabla [],la cual  presenta una comparativa entre las opciones, donde podemos observar que ninguno de los dispositivos logra reunir todas las características deseadas: Un sistema con la capacidad de extraer el estado del sistema y plasmarlo en una plataforma con acceso web, que posea un tiempo de configuración corto, con compatibilidad universal para las diferentes marcas comerciales y con la posibilidad de monitorear más de un elemento dentro del sistema de detección.



\section{Alcance del proyecto}

El presente proyecto implica el diseño e implementación de un sistema de monitoreo remoto a nivel de firmware y hardware, el sistema deberá monitorear al menos una central de alarma de incendio y hasta un máximo de 2 dispositivos secundarios, y definir el estado de la instalación en su totalidad.
El desarrollo se dividirá en dos componentes: el dispositivo primario y el dispositivo secundario y la comunicación entre ambos dispositivos, debe usar tecnología inalámbrica como medio de comunicación 
Este proyecto no incluye la indicación a detalle de los eventos que ocurren en la instalación, el elemento de interés es únicamente el estado de la instalación considerando el dispositivo primario y cada uno de los dispositivos secundarios. Tampoco se considera parte del proyecto el desarrollo de interfaces de usuario, si bien se desarrollan diferentes elementos de interfaz gráfica cumplen con un objetivo de prueba de concepto más que una plataforma final.



\chapter{Introducción específica} % Main chapter title

\label{Chapter2}

%----------------------------------------------------------------------------------------
%	SECTION 1
%----------------------------------------------------------------------------------------

En este capítulo se presentan los requerimientos del sistema y la planificación ejecutada. Además, se detallan las tecnologías utilizadas para el desarrollo tanto de software como de hardware del dispositivo primario y el dispositivo secundario.

\section{Detección de incendio}

Toda central de alarma de incendio debe notificar a los usuarios de la instalación de la presencia de un posible incidente y actuar en concordancia con el protocolo estipulado. Para ello generalmente el sistema segmenta los elementos de entrada y de salida según lo descrito en la figura \ref{fig:figura_a1}. Dividir el sistema facilita las conexiones, mantenimiento y respetar el estándar de la norma; en resumen la división consiste en: 

\begin{itemize}
\item  Lazo de detección: circuito eléctrico que contiene todos aquellos dispositivos asociados específicamente a la detección de posibles focos de incendio.
\item  Lazo de notificación: circuito eléctrico donde se encuentran aquellos equipos asociados a la notificación de un evento de alarma pueden ser parlantes, sirenas, luces estroboscópicas, entre otros.
\item  Interfaz con otros sistemas: relés programables ante diferentes eventos, un conjunto de interruptores que se utilizan para la notificación de eventos. Permite aislar el sistema y a la vez indicar de forma binaria el estado de un parámetro en particular.
\end{itemize}

\begin{figure}[h]
	\centering
	\includegraphics[scale=.45]{./Figures/Capitulo2/Figura_A.png}
	\caption{Diagrama de entradas y salidas de una central de alarma de incendio.}
	\label{fig:figura_a1}
\end{figure}

\section{Hardware}

Los componentes seleccionados para el sistema fueron seleccionados basados en los criterios de la sección \ref{criterios}. Existen tres componentes principales de hardware:  

\begin{itemize}
\item Raspberry Pi (dispositivo primario): es un computador de bajo costo que ejecuta un sistema operativo Linux \cite{RPI}. Tiene el tamaño de una tarjeta de crédito y cuenta con la capacidad de interactuar con componentes electrónicos externos mediante entradas y salidas de propósito general.

\begin{figure}[h]
	\centering
	\includegraphics[scale=.25]{./Figures/Capitulo2/Figura_B.png}
	\caption{Raspberry Pi.}
	\label{fig:figura_b1}
\end{figure}


\item Node-MCU (dispositivo secundario): es un kit de desarrollo de bajo costo basado en el microcontrolador ESP8266 \cite{NODEMCU}. Algunas funcionalidades resaltantes son: regulador de tensión, entradas/salidas de propósito general y un  convertidor analógico digital. Permite el desarrollo de aplicaciones que requieren conexión a Internet de forma rápida, ya que incorpora conectividad Wi-Fi.

\begin{figure}[h]
	\centering
	\includegraphics[scale=.25]{./Figures/Capitulo2/Figura_C.png}
	\caption{Node-MCU V3.0.}
	\label{fig:figura_c1}
\end{figure}

\item Nrf24l01 (comunicación inalámbrica): dispositivo transceptor \cite{rf24} con un protocolo incorporado de comunicación (Enhanced ShockBurst), que permite el diseño de sistemas comunicación inalámbrico con cualquier microcontrolador que cuente con comunicación SPI. Cuenta con parámetros configurables como frecuencia de trabajo (entre 2,400 - 2,4835 GHz), potencia y velocidad de transmisión de datos; lo que lo convierte en un sistema muy flexible para sistemas de bajo que requieran cumplir con requerimientos de bajo consumo.

\begin{figure}[h]
	\centering
	\includegraphics[scale=.65]{./Figures/Capitulo2/Figura_D.png}
	\caption{Nrf24l01+ y Nrf24l01+ con amplificador de potencia.}
	\label{fig:figura_d1}
\end{figure}
\end{itemize}

\section{Protocolos e interfaces}

Esta sección describe los protocolos que se utilizan internamente, junto con las plataformas y diferentes tecnologías empleadas por el sistema.      

\subsection{Protocolos de comunicación }
El sistema cuenta con diferentes equipos que se encuentran intercambiando información constantemente, por lo que se requiere de los siguientes protocolos que regulan la comunicación entre los dispositivos:

\begin{itemize}
\item  SPI (Serial Peripheral Interface): es un protocolo \cite{SPI} que permite la comunicación entre un dispositivo denominado “maestro” y varios dispositivos “esclavos”. El dispositivo maestro es responsable de iniciar la comunicación y define la tasa de transmisión basado en su señal de clock, es utilizado ampliamente entre microcontroladores y es el nexo que permite a los dispositivos primario y secundario interactuar con el transceptor Nrf24l01.

\item  TCP/IP + HTTP (Conexion a Internet ): el dispositivo primario requiere hacer la carga de información a un servidor web, para ello utiliza el protocolo TCP/IP para las capas de transporte y red, en conjunto con el protocolo HTTP para la capa de aplicación del estándar OSI.

\item  Enhanced Shockburst: un protocolo de comunicación basado en paquetes de datos que maneja de forma automática el empaquetado de la información, sincronización, transmisión y reconocimiento automático de transacciones. El protocolo soporta bidireccionalidad y se encuentra embebido en los elementos de la serie Nrf24lxx \cite{nrf24_protocol}.
\end{itemize}

\subsection{Interfaces}

El dispositivo primario otorga la información acerca del estado del sistema  al usuario. Requiere el desarrollo de Interfaces que permitan al usuario identificar fácilmente el estado de la instalación, al igual que un medio de acceso remoto sencillo que le permita acceder en cualquier momento.  

\begin{itemize}
\item Node-Red: herramienta de programación que provee una interfaz de desarrollo para servicios en línea \cite{node_red}. Funciona a través de programación en bloques y cuenta con diferentes integraciones con servicios de Internet, lo que facilita el desarrollo de interfaces de usuario personalizables.    

\item Firebase: es una plataforma de desarrollo de aplicaciones móviles \cite{firebase}. Brinda al usuario servicios de autenticación, bases de datos, almacenamiento, entre otros. Firebase permite realizar todas las conexiones necesarias para exportar la información desde el dispositivo primario a cualquier servicio de Internet, por ejemplo aplicaciones móviles o páginas web. 

\item Bases de datos con SQLite: SQLite es una biblioteca de funcionalidades reducidas que permite generar un motor de búsqueda SQL \cite{sqlite}. Las bases de datos generadas suelen utilizarse como contenedores de información para la transferencia organizada de información entre sistemas. EL uso de bases de datos permite registrar la información generada a partir de los análisis del dispositivo primario del sistema de detección de incendio.

\item MIT App Inventor: es un ambiente de programación en bloque que permite el diseño, simulación y construcción de aplicaciones para equipos móviles. Posee integración con Firebase por lo que con tan solo brindar los códigos de acceso, es posible acceder a la base de datos de cada proyecto.      
\end{itemize}

\section{Requerimientos}

A medida que se desarrollaba el sistema se realizaron reuniones mensuales con consultores y personal técnico de la empresa ISOLSE SRL. Los cambios más significativos consisten en facilitar el conexionado del sistema de monitoreo, eliminar la configuración manual y aprovechar los recursos del panel de control para el suministro eléctrico.   


\subsection{Normas de seguridad y garantía de servicio de Isolse SRL}
	\begin{enumerate}
	\item El sistema no deberá comprometer de ninguna manera el funcionamiento del sistema de detección de incendio.
	\item El sistema no deberá silenciar ni resetear los eventos presentes en el de detección de incendio de forma remota (basados en la norma NFPA punto 23.8.2.10 \cite{nfpa}). 
	\item El sistema de monitoreo remoto no transmitirá fallas internas al sistema de detección.
	\item El sistema de monitoreo deberá instalarse preferentemente en el mismo gabinete que la central de alarma.
	\end{enumerate}
	
\subsection{Funcionalidades comunes entre dispositivos primario y secundario}
	\begin{enumerate}
	\item Se realizará la adquisición del estado de los contactos secos.
		\begin{itemize}
		\item Contacto seco de falla en la central de alarma de incendio.
		\item Contacto seco de alarma en la central de alarma de incendio.
		\end{itemize}
	\item El suministro eléctrico deberá estar dentro del rango de los 12 a 24 VDC.
	\item Se deberá indicar el estado del sistema de detección de incendio a través de Leds, siguiendo la siguientes normas:
		\begin{itemize}
		\item  Verde en normal.
		\item  Amarillo para falla.
		\item  Rojo para alarma.
		\item   Rojo y amarillo para indicar alarma y falla presentes.				
		\end{itemize}
	\end{enumerate}
	
\subsection{Funcionalidades particulares del dispositivo primario}
	\begin{enumerate}
	\item Comunicación servidor web.
		\begin{enumerate}
		\item El sistema deberá tener la capacidad de transmitir información a un servidor web.
		\item La comunicación con el servidor web consistirá en enviar los siguientes datos al servidor web:
			\begin{itemize}
				\item La existencia de fallas en el sistema de detección de incendio. 	
				\item La existencia de alarmas en el sistema de detección de incendio.
				\item La presencia de fallas en el sistema de comunicación inalámbrica.
			\end{itemize}
		\item Se deberá establecer una comunicación con el servidor, ante un evento de cambio de estado del sistema de detección, en un intervalo de tiempo menor a 10 segundos.
		\item Se establecerá una comunicación inmediata con servidor web, si ocurre un cambio en el estado de los contactos secos.
		\item Se deberá establecer una comunicación con el servidor, ante un evento cambio de estado del sistema de comunicación inalámbrica, en un intervalo de tiempo menor a 10 segundos.
		\item Se generará un software encargado de generar y transmitir los paquetes de datos JSON.
		\item Se deberá establecer una prueba que permita verificar el funcionamiento de la comunicación con el servidor web. 
		\end{enumerate}
	\item Se registrarán al menos 500 eventos de forma local.
	\item Comunicación inalámbrica.
		\begin{enumerate}		
		\item Se realizará la adquisición del estado de los dispositivos secundarios a partir de los contactos secos monitoreados:
			\begin{itemize}
				\item Contacto seco de falla en la central de alarma de incendio.
				\item Contacto seco de alarma en la central de alarma de incendio.
			\end{itemize}	
		\item Se debe diseñar un mecanismo de recuperación ante fallas de comunicación inalámbrica.	
		\item Se realizará un software con la capacidad de recibir información por radiofrecuencia con hasta 3 dispositivos secundarios.
		\item Se establecerá una comunicación periódica con todos los dispositivos secundarios con una tasa de refresco de  al menos cada 2,5 s.
		\item Se deberá crear un software para controlar la secuencia de comunicación con dispositivos secundarios.
		\item Se deberá procesar los datos provenientes de dispositivos secundarios, para determinar los estados reales de la instalación.
		\item Se deberá establecer una prueba que permita verificar el funcionamiento de la comunicación por radiofrecuencia.
	\end{enumerate}
\end{enumerate}
\subsection{Funcionalidades particulares del dispositivo secundario}
	\begin{enumerate}	
		\item Se realizará un software con la capacidad de transmitir información por radiofrecuencia al dispositivo primario.
		\item Se establecerá una comunicación periódica con el dispositivo primario con una tasa de refresco de al menos cada 2,5 s.
		\item Se implementará un software con capacidad de reinicio, ante un total de 150 o más intentos fallidos de comunicación inalámbrica.
		\item Se deberá establecer una prueba que permita verificar el funcionamiento de la comunicación inalámbrica.
	\end{enumerate}
	
\subsection{Condiciones de trabajo}
	\begin{enumerate}
		\item Idealmente el sistema tendrá que estar resguardado dentro de la central de incendio, en caso contrario se requerirá una protección IP51.
	\end{enumerate}


\begin{comment}
\begin{enumerate}
\item Normas de seguridad y garantía de servicio de Isolse SRL.
	\begin{enumerate}
	\item El sistema no deberá comprometer de ninguna manera el funcionamiento del sistema de detección de incendio.
	\item El sistema no deberá silenciar ni resetear los eventos presentes en el de detección de incendio de forma remota. 
	\item El sistema de monitoreo remoto no transmitirá fallas internas al sistema de detección.
	\item En caso de corte del suministro eléctrico el sistema no deberá afectar de ninguna manera la autonomía de la central.
	\item El sistema de monitoreo deberá instalarse preferentemente en el mismo gabinete que la central de alarma.
	\end{enumerate}
\item Funcionalidades comunes entre dispositivos primario y secundario.
	\begin{enumerate}
	\item Se realizará la adquisición del estado de los contactos secos.
		\begin{itemize}
		\item Contacto seco de falla en la central de alarma de incendio.
		\item Contacto seco de alarma en la central de alarma de incendio.
		\end{itemize}
	\item El suministro eléctrico deberá estar dentro del rango de los 12 a 24 VDC.
	\item Se deberá indicar el estado del sistema de detección de incendio a través de Leds, siguiendo la siguientes normas:
		\begin{itemize}
		\item  Verde en normal.
		\item  Amarillo para falla.
		\item  Rojo para alarma.
		\item   Rojo y amarillo para indicar alarma y falla presentes.				
		\end{itemize}
	\end{enumerate}
\item Funcionalidades particulares del dispositivo primario.
	\begin{enumerate}
	\item Comunicación servidor web.
		\begin{enumerate}
		\item El sistema deberá tener la capacidad de transmitir información a un servidor web.
		\item La comunicación con el servidor web consistirá en enviar los siguientes datos al servidor web:
			\begin{itemize}
				\item La existencia de fallas en el sistema de detección de incendio. 	
				\item La existencia de alarmas en el sistema de detección de incendio.
				\item La presencia de fallas en el sistema de comunicación inalámbrica.
			\end{itemize}
		\item Se deberá establecer una comunicación con el servidor, ante un evento de cambio de estado del sistema de detección, en un intervalo de tiempo menor a 10 segundos.
		\item Se establecerá una comunicación inmediata con servidor web, si ocurre un cambio en el estado de los contactos secos.
		\item Se deberá establecer una comunicación con el servidor, ante un evento cambio de estado del sistema de comunicación inalámbrica, en un intervalo de tiempo menor a 10 segundos.
		\item Se generará un software encargado de generar y transmitir los paquetes de datos JSON.
		\item Se deberá establecer una prueba que permita verificar el funcionamiento de la comunicación con el servidor web. 
		\end{enumerate}
	\item Se registrarán al menos 500 eventos de forma local.
	\item Comunicación inalámbrica.
		\begin{enumerate}		
		\item Se realizará la adquisición del estado de los dispositivos secundarios a partir de los contactos secos monitoreados:
			\begin{itemize}
				\item Contacto seco de falla en la central de alarma de incendio.
				\item Contacto seco de alarma en la central de alarma de incendio.
			\end{itemize}	
		\item Se debe diseñar un mecanismo de recuperación ante fallas de comunicación inalámbrica.	
		\item Se realizará un software con la capacidad de recibir información por radiofrecuencia con hasta 3 dispositivos secundarios.
		\item Se establecerá una comunicación periódica con todos los dispositivos secundarios con una tasa de refresco de  al menos cada 2,5 s.
		\item Se deberá crear un software para controlar la secuencia de comunicación con dispositivos secundarios.
		\item Se deberá procesar los datos provenientes de dispositivos secundarios, para determinar los estados reales de la instalación.
		\item Se deberá establecer una prueba que permita verificar el funcionamiento de la comunicación por radiofrecuencia.
	\end{enumerate}
\end{enumerate}
\item Funcionalidades particulares del dispositivo secundario.
	\begin{enumerate}	
		\item Se realizará un software con la capacidad de transmitir información por radiofrecuencia al dispositivo primario.
		\item Se establecerá una comunicación periódica con el dispositivo primario con una tasa de refresco de al menos cada 2,5 s.
		\item Se implementará un software con capacidad de reinicio, ante un total de 150 o más intentos fallidos de comunicación inalámbrica.
		\item Se deberá establecer una prueba que permita verificar el funcionamiento de la comunicación inalámbrica.
	\end{enumerate}
\item Condiciones de trabajo.
	\begin{enumerate}
		\item Idealmente el sistema tendrá que estar resguardado dentro de la central de incendio, en caso contrario se requerirá una protección IP51.
	\end{enumerate}
\end{enumerate}
\end{comment}
 
%\subsection{Uso del Nrf24l01+}
%\begin{itemize}
%\item  Nrf24l01 amplificador de potencia con reducción de ruido y antena.
%\item  Nrf24l01 + anten
%\item  Fuente de poder Canakit micro USB 2,5 A con filtro de ruido.
%\item Adaptador 120-240 VA a 5V DC USB.
%\end{itemize}

%\begin{figure}[ht]
%	\centering
%	\includegraphics[scale=.45]{./Figures/Capitulo4/Figura_A.png}
%	\caption{Esquema utilizado para el ensayo del Nrf24l01+.}
%	\label{fig:figura_a}
%\end{figure}
%\ref{fig:figura_a}
 
\chapter{Diseño e implementación} % Main chapter title

\label{Chapter3} % Change X to a consecutive number; for referencing this chapter elsewhere, use \ref{ChapterX}

\definecolor{mygreen}{rgb}{0,0.6,0}
\definecolor{mygray}{rgb}{0.5,0.5,0.5}
\definecolor{mymauve}{rgb}{0.58,0,0.82}

%%%%%%%%%%%%%%%%%%%%%%%%%%%%%%%%%%%%%%%%%%%%%%%%%%%%%%%%%%%%%%%%%%%%%%%%%%%%%
% parámetros para configurar el formato del código en los entornos lstlisting
%%%%%%%%%%%%%%%%%%%%%%%%%%%%%%%%%%%%%%%%%%%%%%%%%%%%%%%%%%%%%%%%%%%%%%%%%%%%%
\lstset{ %
  backgroundcolor=\color{white},   % choose the background color; you must add \usepackage{color} or \usepackage{xcolor}
  basicstyle=\footnotesize,        % the size of the fonts that are used for the code
  breakatwhitespace=false,         % sets if automatic breaks should only happen at whitespace
  breaklines=true,                 % sets automatic line breaking
  captionpos=b,                    % sets the caption-position to bottom
  commentstyle=\color{mygreen},    % comment style
  deletekeywords={...},            % if you want to delete keywords from the given language
  %escapeinside={\%*}{*)},          % if you want to add LaTeX within your code
  %extendedchars=true,              % lets you use non-ASCII characters; for 8-bits encodings only, does not work with UTF-8
  %frame=single,	                % adds a frame around the code
  keepspaces=true,                 % keeps spaces in text, useful for keeping indentation of code (possibly needs columns=flexible)
  keywordstyle=\color{blue},       % keyword style
  language=[ANSI]C,                % the language of the code
  %otherkeywords={*,...},           % if you want to add more keywords to the set
  numbers=left,                    % where to put the line-numbers; possible values are (none, left, right)
  numbersep=5pt,                   % how far the line-numbers are from the code
  numberstyle=\tiny\color{mygray}, % the style that is used for the line-numbers
  rulecolor=\color{black},         % if not set, the frame-color may be changed on line-breaks within not-black text (e.g. comments (green here))
  showspaces=false,                % show spaces everywhere adding particular underscores; it overrides 'showstringspaces'
  showstringspaces=false,          % underline spaces within strings only
  showtabs=false,                  % show tabs within strings adding particular underscores
  stepnumber=1,                    % the step between two line-numbers. If it's 1, each line will be numbered
  stringstyle=\color{mymauve},     % string literal style
  tabsize=2,	                   % sets default tabsize to 2 spaces
  title=\lstname,                  % show the filename of files included with \lstinputlisting; also try caption instead of title
  morecomment=[s]{/*}{*/}
}


%----------------------------------------------------------------------------------------
%	SECTION 1
%----------------------------------------------------------------------------------------

En este capítulo se exponen los detalles del diseño de los dispositivos primario y secundarios, se describe el desarrollo y funcionamiento del hardware, software y las características más resaltantes del proyecto.

\section{Sistemas complejos de detección de incendios}

Un sistema de detección de incendio no se limita a un único edificio, en instalaciones de mayor escala es común observar sistemas compuestos por más de un edificio, donde se requiere del monitoreo de áreas comunes o incluso zonas que son de particular interés para el tópico de alarmas de incendio, como por ejemplo depósito de materiales inflamables, generadores eléctricos de respaldo o sistema de bombeo de extinción.

En este nivel los sistemas suelen estar compuestos por más de una central de alarma de incendio, las cuales pueden interconectadas entre sí y formar una red de dispositivos de detección, dependiendo del nivel de integración, los sistemas pueden tener total independencia o por el contrario funcionar como una única unidad de detección, por lo que  no siempre es una tarea sencilla identificar la ubicación de origen de un evento.

Un ejemplo de una instalación con potencial para un sistema de detección de incendio complejo puede observarse en la figura \ref{fig:figura_a3}, donde podemos observar una propiedad compuesta por tres edificios y una zona crítica, con una posible estructura de dispositivos primarios y secundarios que permite definir el estado del sistema de detección de incendio en su totalidad.


\begin{figure}[]
	\centering
	\includegraphics[scale=.25]{./Figures/Capitulo3/Fig_A3.png}
	\caption{Ejemplo de sistema de detección de incendio complejo.}
	\label{fig:figura_a3}
\end{figure}

\section{Criterios de diseño}

El sistema corresponde al primero de una serie de proyectos que tienen como objetivo la generación de alternativas de monitoreo de sistemas de detección de incendio, en esta etapa el objetivo propuesto es adquirir únicamente el estado de la instalación, pero en futuros proyectos se desea alcanzar un mayor nivel de detalle, a continuación se describen los criterios considerados para la selección de los componentes:

Escalabilidad: El sistema debe contar con los recursos necesarios para la integración de nuevas funcionalidades o servicios.

Robustez: Una característica que se desea proporcionar a los sistemas es la posibilidad de tener formas alternativas de trabajo ante eventos de falla.

Recuperabilidad: El mantenimiento de equipos de detección de incendio suele realizarse de forma mensual, por lo que el sistema debe estar en capacidad de restituirse de forma automática en caso de fallas, para evitar visitas adicionales por fallas del dispositivo de monitoreo.

Documentación adecuada:  El sistema debe desarrollarse a partir de plataformas con documentación detallada y de fácil acceso.

Disponibilidad en el mercado Argentino: Los sistemas deberán estar compuestos por elementos que puedan ser adquiridos en el mercado local.

\subsubsection{Criterios fundamentales}

Es de suma importancia respetar dos criterios principales:
\begin{itemize}
\item La no afectación la secuencia de accionamiento de la central de forma remota. 
\item El sistema de monitoreo no debe utilizarse como un dispositivo de detección de incendio.
\end{itemize}

En la figura \ref{fig:figura_b3} podemos observar esquemas de conexión y el uso recomendado del sistema de monitoreo.

\begin{figure}[]
	\centering
	\includegraphics[scale=.25]{./Figures/Capitulo3/Fig_B3.png}
	\caption{Esquema de conexión entre dispositivo de monitoreo y central de alarma de incendio.}
	\label{fig:figura_b3}
\end{figure}

\section{Arquitectura general del sistema}

En esta fase se procede a describir los elementos que componen el sistema de monitoreo, su funcionamiento y la interconexión entre los mismos.


\subsection{Dispositivo primario} 

El dispositivo primario cuenta con los recursos necesarios para el monitoreo de un sistema de detección de alarma base, compuesto por una única central de alarma de incendio a monitorear. El dispositivo implementa un monitoreo periodico a los contactos de alarma y falla para determinar el estado del equipo local, el siguiente paso es facilitar esta información al usuario, para lo que hace uso de tres recursos de notificación:
\begin{itemize}
\item Leds de indicación de estado.
\item Interfaz web para notificación local, por medio de la plataforma Node-RED.
\item Servidor web firebase.
\end{itemize}

A pesar de ser un dispositivo para el monitoreo de un sistema de detección de alarma base, este dispositivo considera la posibilidad de un sistema de detección complejo, para lo que dispone de un módulo específico para la gestión de comunicación inalámbrica, que permite al sistema comunicarse con diferentes dispositivos de monitoreo secundarios, cuyo objetivo principal es reportar el estado de cada subsistema o equipo que se desee monitorear.

Si el dispositivo se va emplear para sistemas complejos de detección, el dispositivo primario procesa el estado local y los estados remotos para generar un diagnóstico del sistema de detección en general, de esta forma en caso de no existir vinculación entre los equipos de detección de alarma el sistema es capaz de establecer el estado correcto de la instalación en su totalidad.

En la figura \ref{fig:figura_c3} se presenta el diseño general del dispositivo primario. 

\begin{figure}[]
	\centering
	\includegraphics[scale=.25]{./Figures/Capitulo3/Fig_C3.png}
	\caption{Arquitectura del dispositivo primario.}
	\label{fig:figura_c3}
\end{figure}

\subsection{Dispositivo secundario}

El dispositivo secundario realiza el  monitoreo de los contactos de alarma y falla, cuenta con módulo destinado a la gestión de la transmisión de información mediante comunicación inalámbrica. El sistema presenta la información al usuario mediante tres leds de colores específicos, rojo para alarma, amarillo para fallas y verde para normal.


Incluir en el diseño un dispositivo secundario permite incluir un equipo adicional a monitorear, y además incrementa el alcance de la comunicación inalámbrica, ya que cada nodo actúa como nodo-repetidor. En la figura \ref{fig:figura_d3} se observa el diseño general del dispositivo secundario.


\begin{figure}[]
	\centering
	\includegraphics[scale=.25]{./Figures/Capitulo3/Fig_D3.png}
	\caption{Arquitectura del dispositivo secundario.}
	\label{fig:figura_d3}
\end{figure} 

\subsection{Aplicación en sistemas de detección complejos}

Una instalación como la descrita en la figura \ref{fig:figura_a3}, compuesto por tres edificios y una zona crítica, puede ser monitoreado con una estructura como la dispuesta en la figura \ref{fig:figura_e3}, donde podemos observar las interfaces entre los elementos del sistema,  y cómo se relacionan entre ellos.

\begin{figure}[]
	\centering
	\includegraphics[scale=.25]{./Figures/Capitulo3/Fig_E3.png}
	\caption{Arquitectura general del dispositivo de monitoreo remoto.}
	\label{fig:figura_e3}
\end{figure} 

\section{Hardware}

Esta sección describe el  diseño del hardware utilizado para los dispositivos primario y secundario, sus características, módulos internos y la relación entre ellos.

\subsection{Hardware del dispositivo primario}

Las figuras \ref{fig:figura_f3}, \ref{fig:figura_g3} y \ref{fig:figura_h3}, muestran los esquemáticos empleados para la implementación del hardware del dispositivo primario. El diseño fue concretado como un poncho para la Raspberry Pi, compuesto por los siguientes módulos:

\subsubsection{Módulo de monitoreo de contactos secos}

El diseño de este módulo hace uso de los contactos normal abierto del dispositivo a monitorear para la indicación de alarma o falla en el sistema. La indicación correspondiente se hace a través del accionamiento de un relé interno que modifica el estado del circuito asociado.
La figura \ref{fig:figura_f3} muestra el esquema del módulo de monitoreo de contacto secos.

\begin{figure}[]
	\centering
	\includegraphics[scale=.25]{./Figures/Capitulo3/Fig_F3.png}
	\caption{Módulo de monitoreo de contacto seco de alarma.}
	\label{fig:figura_f3}
\end{figure} 

El diseño considera la posibilidad de ocurrencia de un error de conexión, para lo cual se incluyó una etapa de protección, que se encarga de mantener la tensión dentro de los rangos de trabajo del dispositivo [0 - 12] vdc.

\subsubsection{Módulo de comunicación nrf24l01+.}

La conexión del módulo transceptor nrf24l01+ se hace siguiendo el esquema recomendado [bibliografía], como puede observarse en la figura \ref{fig:figura_g3}, consiste en una conexión directa con el dispositivo y alimentado por la fuente de 3.3 vdc del dispositivo primario.  

\begin{figure}[]
	\centering
	\includegraphics[scale=.25]{./Figures/Capitulo3/Fig_G3.png}
	\caption{Módulo de comunicación inalámbrica.}
	\label{fig:figura_g3}
\end{figure} 

\subsubsection{Módulo de notificación visual}

La notificación visual sigue lo establecido por los requerimientos  ......... y permite al  dispositivo reportar el estado del sistema a monitorear de forma local, sin necesidad de dispositivos adicionales. En la figura \ref{fig:figura_h3} se presentan las conexiones y pines utilizados por el sistema.

\begin{figure}[]
	\centering
	\includegraphics[scale=.25]{./Figures/Capitulo3/Fig_H3.png}
	\caption{Módulo de notificación visual.}
	\label{fig:figura_h3}
\end{figure} 


\subsection{Hardware del dispositivo secundario}

El diseño del sistema puede observarse en las figuras \ref{fig:figura_j3},\ref{fig:figura_k3},\ref{fig:figura_l3},\ref{fig:figura_m3}, se reutiliza el diseño de la figura \ref{fig:figura_f3} para el monitoreo de contactos y añaden al sistema los módulos descritos a continuación.

\subsubsection{Módulo de comunicación nrf24l01+ con amplificador de potencia}

El dispositivo secundario utiliza un módulo nrf24l01+ con un amplificador de potencia, por lo que a diferencia del dispositivo primario, se utiliza un adaptador para la conexión del módulo que garantiza una tensión de alimentación estable. El esquema de conexión se detalla en la figura \ref{fig:figura_j3}.

\begin{figure}[]
	\centering
	\includegraphics[scale=.25]{./Figures/Capitulo3/Fig_J3.png}
	\caption{Módulo de comunicación inalámbrica de largo alcance.}
	\label{fig:figura_j3}
\end{figure} 


\subsubsection{Módulo de notificación visual}

Debido a restricciones del dispositivo con respecto al número de GPIOS disponibles, el sistema no implementa una conexión directa con el led de notificación de estado normal, en la figura \ref{fig:figura_k3} se puede verificar que el encendido del led indicador del estado normal, se realiza a través de hardware ante la ausencia de señales de falla o alarma.

\begin{figure}[]
	\centering
	\includegraphics[scale=.25]{./Figures/Capitulo3/Fig_K3.png}
	\caption{Módulo de notificación visual.}
	\label{fig:figura_k3}
\end{figure} 


\subsubsection{Módulo de alimentación}

La alimentación del dispositivo se obtiene a partir del módulo LM2596S-123-3000, un regulador de tensión continua regulable, configurado a 12 vdc. La figura \ref{fig:figura_l3} exhibe al módulo de alimentación y el diseño de una etapa de protección ante casos de conexión con polaridad inversa.

\begin{figure}[]
	\centering
	\includegraphics[scale=.25]{./Figures/Capitulo3/Fig_L3.png}
	\caption{Módulo de notificación visual.}
	\label{fig:figura_k3}
\end{figure} 


\section{Software}


% Chapter Template

\chapter{Ensayos y Resultados} % Main chapter title

\label{Chapter4} % Change X to a consecutive number; for referencing this chapter elsewhere, use \ref{ChapterX}

%----------------------------------------------------------------------------------------
%	SECTION 1
%----------------------------------------------------------------------------------------

Este capítulo contiene la  descripción de  las pruebas realizadas para la validación del sistema. La metodología para esta fase partió desde un nivel micro en módulos funcionales,  hasta alcanzar el nivel de pruebas de sistema,  y se finalizó con ensayos de inserción de fallas.


\section{Validación de componentes}
\label{sec:validacion_componentes}

Como se explicó anteriormente, los criterios escogidos para la selección de los dispositivos fueron: a) el factor económico, nivel de documentación,  c) disponibilidad en el mercado argentino, d) tiempo estimado de desarrollo. Sin embargo, el criterio determinante se basó en los resultados generados en los ensayos que se describen a continuación.

\subsection{Nrf24l01+}

Este ensayo fue realizado como un módulo independiente, con el objetivo de evaluar el desempeño del transceptor nrf24l01 bajo condiciones ideales y su estabilidad al utilizar la biblioteca seleccionada.

La prueba consistió  en establecer la comunicación inalámbrica entre dos nodos separados a una distancia de 40 cm. Uno de los nodos mantiene un conteo incremental e intenta transmitir la última lectura al próximo nodo -el cual se mantiene a la espera del mensaje- en caso de que transcurra un tiempo mayor o igual a 200 ms sin recibir un mensaje, se registra un evento de timeout. El resultado se obtiene a partir del porcentaje de mensajes erróneos en relación a los totales enviados, durante un periodo de tiempo de una hora.

%\textit{testing}.
Materiales utilizados:
\begin{itemize}
\item  Nrf24l01 amplificador de potencia con reducción de ruido y antena.
\item  Nrf24l01 + antena PCB.
\item  Arduino Uno.   
\item  NodeMCU v3.
\item  Adafruit Feather HUZZAH ESP8266.
\item  Raspberry Pi.
\item  Fuente de poder Canakit micro USB 2,5 A con filtro de ruido.
\item Adaptador 120-240 VA a 5V DC USB.
\end{itemize}

La figura \ref{fig:figura_a} muestra un esquema de la metodología empleada para realizar la primera prueba de este ensayo; la Raspberry Pi se seleccionó como el único nodo fijo durante estos ensayos, el resto de dispositivos adoptará sucesivamente el rol de nodo secundario uno a la vez. Los resultados de esta prueba son bastante desalentadores, ya que para cada dispositivo el porcentaje de mensajes exitosos apenas alcanza un 0,5 \% en el mejor de los casos.


\begin{figure}[ht]
	\centering
	\includegraphics[scale=.45]{./Figures/Capitulo4/Figura_A.png}
	\caption{Esquema utilizado para el ensayo del Nrf24l01+.}
	\label{fig:figura_a}
\end{figure}


Se inició una fase de búsqueda de posibles causas, que inició por una revisión de ejemplos de uso del dispositivo según la documentación de la biblioteca, así como la revisión de pruebas con resultados similares de diferentes desarrolladores y su experiencia reportada. En función de esto se identificó que un elemento clave, es proveer al dispositivo de una fuente de alimentación estable, ya que en algunos casos presentaban fallas similares causadas por fluctuaciones y ruido presente en la alimentación del dispositivo.

Una vez identificada una posible solución, se realizó nuevamente el ensayo reemplazando el adaptador por un equipo de iguales características pero de mejor calidad. Los resultados de esta segunda prueba se consideran satisfactorios y están representados en la figura \ref{fig:figura_b}, donde se puede apreciar que el menor valor corresponde a un 98 \% de éxito en mensajes transmitidos.


\begin{figure}[ht]
	\centering
	\includegraphics[scale=.45]{./Figures/Capitulo4/Figura_B.png}
	\caption{Resultados ensayo del Nrf24l01+.}
	\label{fig:figura_b}
\end{figure}

Estos resultados evidencian lo susceptible que es el sistema a ruido en la alimentación y la necesidad de una fuente regulada, estable, con el menor ruido posible, es por esto que a partir de este punto se incluye en el diseño del sistema el adaptador del módulo transceptor nrf24l01+. Una característica a resaltar de este adaptador es que imposibilita el uso del nrf24l01+ convencional de antena en circuito impreso, es compatible únicamente con el modelo con amplificador de potencia y reducción de ruido con antena.


\subsection{Biblioteca RF24}
 

El objetivo de este ensayo es verificar el uso de la biblioteca RF24 para la transmisión de paquetes de información haciendo uso del transceptor nrf24l01+. La disposición de los equipos utilizados para este ensayo consiste en:

\begin{itemize}
\item Utilizar la Raspberry Pi junto con un transceptor nrf24l01 + de antena interna como dispositivo primario de ubicación fija.
\item Un dispositivo secundario móvil, compuesto por el Arduino UNO y nrf24l01+ de antena interna.
\end{itemize}

El objetivo de este ensayo fue establecer el alcance de los dispositivos bajo las siguientes condiciones en común: 
\begin{itemize}
\item Presencia de redes WiFi de diferente intensidad.
\item Equipos electrónicos de bajo consumo.
\end{itemize}

Las condiciones antes mencionadas se mantuvieron a lo largo del ensayo y adicional a esto, se procedió a evaluar la respuesta ante situaciones con los siguientes tipos de obstrucción:  
\begin{itemize}
\item Obstrucción nula.
\item Obstrucción total.
	\begin{itemize}
	\item  Puerta de vidrio
	\item  Puerta metálica.
	\end{itemize} 
\end{itemize}  

El propósito de la prueba es medir la máxima distancia que puede existir entre los nodos primario y secundario, manteniendo una comunicación inalámbrica estable con una tasa de éxito de al menos 90 \% de mensajes transmitidos. En la figura \ref{fig:figura_c} se puede apreciar la metodología utilizada para el ensayo, inicialmente se parte de una distancia se confirmaba que la comunicación entre los dispositivos era correcta y se procedía mover el dispositivo secundario a una nueva ubicación y repetir nuevamente el proceso de verificación, así hasta alcanzar la máxima distancia a la que era posible establecer la comunicación entre los dispositivos. 

\begin{figure}[ht]
	\centering
	\includegraphics[scale=.3]{./Figures/Capitulo4/Figura_C.png}
	\caption{Esquema del ensayo de la biblioteca RF24.}
	\label{fig:figura_c}
\end{figure}

Los resultados se muestran en la tabla \ref{tab:tabla_1}, donde se puede apreciar la susceptibilidad del sistema ante la obstrucción de objetos sólidos.

\begin{table}[h]
\centering
\caption[Resultados ensayo biblioteca RF24]{Resultados de alcance logrado para el ensayo de la biblioteca RF24}
\begin{tabular}{lccc}
\toprule
\textbf{}                       & \multirow{2}{*}{\textbf{Sin obstáculos}} & \multicolumn{2}{c}{\textbf{Obstrucción total}} \\
\textbf{}                       &                                          & \textbf{Ventanal}  & \textbf{Puerta metálica}  \\
\midrule
\multicolumn{1}{c}{Alcance (m)} & 8,5                                      & 8                  & 0                        
\\
\bottomrule
\hline                                                                        
\end{tabular}
\label{tab:tabla_1}
\end{table}


\subsection{Biblioteca RF24Mesh}

El ensayo anterior establece los valores aproximados de alcance en diferentes condiciones, para este ensayo se requiere validar el funcionamiento de la biblioteca RF24Mesh. 
El objetivo del ensayo se mantiene, registrar la máxima distancia que se puede alcanzar al separar los dispositivos, manteniendo una comunicación estable. Las condiciones de la prueba son similares al ensayo anterior, consiste en dos nodos en ubicaciones fijas separados una distancia de 7 m, pero esta vez se añade un segundo dispositivo secundario que se utilizará para medir el alcance en esta nueva configuración.


Los resultados del ensayo se pueden observar en la figura \ref{fig:figura_d}], se obtuvo una distancia máxima de 33 m, al comparar con los resultados del ensayo nrf24l01, se aprecia un incremento en el alcance del sistema, pero esto es debido a las características del dispositivo secundario 2, ya que cuenta con un sistema de amplificación de potencia.

\begin{figure}[ht]
	\centering
	\includegraphics[scale=.3]{./Figures/Capitulo4/Figura_D.png}
	\caption{Resultados del ensayo de la biblioteca RF24Mesh.}
	\label{fig:figura_d}
\end{figure}

Adicional a la medición del alcance en la nueva configuración de dispositivos inalámbricos, se validan las siguientes funcionalidades de la biblioteca RF24Mesh:


\begin{itemize}
\item Gestión automática de dispositivos conectados: el sistema fue capaz de incluir de manera automática al dispositivo secundario 2 sin necesidad de configuraciones adicionales.
\item Retransmisión de mensajes: El sistema cuenta con la posibilidad de retransmitir los mensajes entre los nodos hasta alcanzar el nodo destino, esto se comprueba ya que no fue posible establecer la comunicación de forma directa entre el dispositivo secundario 2 y el dispositivo primario, pero al existir el dispositivo secundario 1 en la red, este se encarga de recibir el mensaje del dispositivo secundario 2 y transmitirlo al dispositivo primario.
\end{itemize}


\section{Hardware}

Partiendo de los requerimientos del sistema, se realizó la fabricación de placas prototipo, para la verificación de los diseños de los módulos que componen el del sistema. A continuación se presentan los resultados obtenidos del diseño y fabricación de hardware.  


\subsection{Prototipo: dispositivo primario.}


\subsection{Prototipo: dispositivo secundario.}

El prototipo se basa en los esquemáticos de las figuras \ref{fig:figura_1},\ref{fig:figura_2},\ref{fig:figura_3}, cada uno de estos diseños fue probado mediante simulaciones realizadas, en la plataforma proteus. En la figura \ref{fig:figura_f} se presenta el prototipo realizado, a pesar de no ser un dispositivo secundario completo, su elaboración fue muy enriquecedora para el diseño final, influyó en una reducción importante de costos de producción tanto económicos como de tiempo invertido.  

inicialmente el objetivo de la fabricación del prototipo era permitir realizar pruebas reales del software durante toda la etapa de desarrollo del trabajo, pero adicional a los aportes de \textit{testing} de software, la elaboración proporcionó la información necesaria para detectar puntos de falla desapercibidos hasta el momento.

\begin{figure}[ht]
	\centering
	\includegraphics[scale=.45]{./Figures/Capitulo4/pendiente.jpg}
	\caption{pendiente}
	\label{fig:figura_1}
\end{figure}
\begin{figure}[ht]
	\centering
	\includegraphics[scale=.45]{./Figures/Capitulo4/pendiente.jpg}
	\caption{pendiente}
	\label{fig:figura_2}
\end{figure}
\begin{figure}[ht]
	\centering
	\includegraphics[scale=.45]{./Figures/Capitulo4/pendiente.jpg}
	\caption{pendiente}
	\label{fig:figura_3}
\end{figure}
\begin{figure}[ht]
	\centering
	\includegraphics[scale=.45]{./Figures/Capitulo4/pendiente.jpg}
	\caption{pendiente}
	\label{fig:figura_f}
\end{figure}
\begin{figure}[ht]
	\centering
	\includegraphics[scale=.3]{./Figures/Capitulo4/pendiente.jpg}
	\caption{pendiente}
	\label{fig:figura_1}
\end{figure}


A simple vista se puede observar en la figura \ref{fig:figura_e} que el NodeMCU cuenta con GPIOS suficientes para la implementación del dispositivo secundario, ya que el número de recursos disponibles supera el número de recursos requeridos. El diseño descrito en el capítulo \ref{Chapter3} ejecuta un protocolo de reset automático, como parte de un mecanismo de  recuperación ante fallas en la comunicación inalámbrica, eléctricamente el disparador de este mecanismo es un estado digital alto en el pin RST, lo que implica, que es de vital importancia que el pin RST se mantenga en estado digital bajo en todo momento, hasta ser requerido por el algoritmo.

\begin{figure}[ht]
	\centering
	\includegraphics[scale=.45]{./Figures/Capitulo4/Figura_E.png}
	\caption{Resumen de GPIOS de interés para la implementación el sistema}
	\label{fig:figura_e}
\end{figure}

Durante el encendido del NodeMCU, se ejecuta una secuencia de arranque que altera el estado de diferentes GPIOS de forma aleatoria, la tabla \ref{tab:tabla_2} hace referencia a esta secuencia de arranque y muestra un listado de los pines afectados durante el arranque del chip. Al cruzar la información de la figura \ref{fig:figura_e}, con los pines que cumplen la condición de estabilidad durante el arranque según la tabla \ref{tab:tabla_2} podemos inferir que la aseveración realizada anteriormente no es correcta, el número de recursos solicitados es mayor al número de recursos disponibles, lo que hace imposible la implementación del diseño haciendo uso exclusivo de entradas y salidas digitales.

%tabla_2
\begin{table}[h]
\centering
\caption[GPIOS NodeMCU]{Descripción de pines durante arranque de NodeMCU}
\begin{tabular}{ccccc}
\toprule
\textbf{Etiqueta} & \textbf{GPIO} & \textbf{Input}                                               & \textbf{Output} & \textbf{Secuencia de arranque}                                                                 \\
\midrule
D0                & 16            & Sin problema                                                 & Sin problema    & Alto al arranque                                                                               \\
D1                & 5             & Sin problema                                                 & Sin problema    & -                                                                                              \\
D2                & 4             & Sin problema                                                 & Sin problema    & -                                                                                              \\
D3                & 0             & \begin{tabular}[c]{@{}c@{}}Conexión\\ pull up\end{tabular}   & Sin problema    & \begin{tabular}[c]{@{}c@{}}El arranque falla \\ si se encuentra en\\  estado bajo\end{tabular} \\
D4                & 2             & \begin{tabular}[c]{@{}c@{}}Conexión\\ pull up\end{tabular}   & OK              & Alto al arranque                                                                               \\
D5                & 14            & SPI                                                          & SPI             & -                                                                                              \\
D6                & 12            & SPI                                                          & SPI             & -                                                                                              \\
D7                & 13            & SPI                                                          & SPI             & -                                                                                              \\
D8                & 15            & \begin{tabular}[c]{@{}c@{}}Conexión\\ pull down\end{tabular} & OK              & \begin{tabular}[c]{@{}c@{}}El arranque falla\\ si se encuentra en\\ estado alto\end{tabular}   \\
RX                & 3             & UART                                                         & UART            & Alto al arranque                                                                               \\
TX                & 1             & UART                                                         & UART            & Alto al arranque                                                                               \\
A0                & ADC0          & \begin{tabular}[c]{@{}c@{}}Entrada\\ analogica\end{tabular}  & -               & -                                                                                             
\\
\bottomrule
\hline                                                                        
\end{tabular}
\label{tab:tabla_2}
\end{table}


La solución a este problema radica en aprovechar otra característica del NodeMCU, la medición del voltaje se realizará a través del convertidor analógico digital incluido en el microcontrolador, lo que permite compensar el déficit de GPIOS y mantener el diseño intacto, en concreto se utilizará para el monitoreo del estado del contacto seco de falla.



\subsection{Prototipo: dispositivo secundario.}

Al recopilar la información obtenida de la fabricación de los prototipos, el siguiente paso consistió en consolidar todas estas consideraciones en el esquemático del sistema, para dar inicio a la fase de elaboración de dispositivos para la fabricación, las figuras \ref{fig:figura_h} e \ref{fig:figura_i} presentan el circuito impreso generado en vista 3D y finalmente la placa en su estado final.

\begin{figure}[ht]
	\centering
	\includegraphics[scale=.25]{./Figures/Capitulo4/Figura_H.png}
	\caption{Vista 3d del diseño de la placa del dispositivo secundario.}
	\label{fig:figura_h}
\end{figure}

\begin{figure}[ht]
	\centering
	\includegraphics[scale=.3]{./Figures/Capitulo4/pendiente.jpg}
	\caption{pendiente}
	\label{fig:figura_i}
\end{figure}


\section{Pruebas unitarias}

\subsection{Estado local.}

La tabla \ref{tab:tabla_3} corresponde a los casos de prueba obtenidos como resultado de aplicar la metodología CTM, apoyándonos en el diagrama de la figura \ref{fig:figura_j} como punto de partida. 
        
   
Al comparar los resultados obtenidos con los resultados esperados, se concluye que el dispositivo primario funciona correctamente, por lo que a partir de la modificación de los contactos de alarma o falla, somos capaces de establecer correctamente el estado del sistema de monitoreo.


\begin{table}[h]
\centering
\caption[Casos de prueba, estado local]{Casos de prueba del ensayo de sistema con un dispositivo primario}
\begin{tabular}{clcllc}
\toprule
\textbf{\begin{tabular}[c]{@{}c@{}}Caso de\\ prueba\end{tabular}} & \multicolumn{1}{c}{\textbf{Parámetro}}                                        & \textbf{Valor}           & \multicolumn{2}{c}{\textbf{Comentario}}                                                                                                                                                              & \textbf{\begin{tabular}[c]{@{}c@{}}Resultado\\ esperado\end{tabular}}                                                    \\
\midrule
\multirow{5}{*}{1}                                                & \begin{tabular}[c]{@{}l@{}}Contacto\\ de alarma\end{tabular}                  & Cerrado                  & \multicolumn{2}{l}{\multirow{5}{*}{\begin{tabular}[c]{@{}l@{}}Sistema monitoreado en\\ estado de alarma y falla,\\ al menos una alarma y\\ una falla están presentes\\ en el sistema.\end{tabular}}} & \multirow{5}{*}{\begin{tabular}[c]{@{}c@{}}Led:\\ rojo-amarillo\\ Estado RF:\\ ok\\ Estado:\\ alarma-falla\end{tabular}} \\
                                                                  & \multirow{4}{*}{\begin{tabular}[c]{@{}l@{}}Contacto\\ de falla\end{tabular}}  & \multirow{4}{*}{Cerrado} & \multicolumn{2}{l}{}                                                                                                                                                                                 &                                                                                                                          \\
                                                                  &                                                                               &                          & \multicolumn{2}{l}{}                                                                                                                                                                                 &                                                                                                                          \\
                                                                  &                                                                               &                          & \multicolumn{2}{l}{}                                                                                                                                                                                 &                                                                                                                          \\
                                                                  &                                                                               &                          & \multicolumn{2}{l}{}                                                                                                                                                                                 &                                                                                                                          \\
\multicolumn{1}{l}{}                                              &                                                                               & \multicolumn{1}{l}{}     &                                                                                                   &                                                                                                  & \multicolumn{1}{l}{}                                                                                                     \\
\midrule
\multirow{5}{*}{2}                                                & \begin{tabular}[c]{@{}l@{}}Contacto\\ de alarma\end{tabular}                  & Abierto                  & \multicolumn{2}{l}{\multirow{5}{*}{\begin{tabular}[c]{@{}l@{}}Sistema monitoreado en\\ estado normal, sin fallas\\ ni alarmas presentes.\end{tabular}}}                                              & \multirow{5}{*}{\begin{tabular}[c]{@{}c@{}}Led:\\ verde\\ Estado RF:\\ ok\\ Estado:\\ normal\end{tabular}}               \\
                                                                  & \multirow{4}{*}{\begin{tabular}[c]{@{}l@{}}Contacto\\ de falla\end{tabular}}  & \multirow{4}{*}{Abierto} & \multicolumn{2}{l}{}                                                                                                                                                                                 &                                                                                                                          \\
                                                                  &                                                                               &                          & \multicolumn{2}{l}{}                                                                                                                                                                                 &                                                                                                                          \\
                                                                  &                                                                               &                          & \multicolumn{2}{l}{}                                                                                                                                                                                 &                                                                                                                          \\
                                                                  &                                                                               &                          & \multicolumn{2}{l}{}                                                                                                                                                                                 &                                                                                                                          \\
\multicolumn{1}{l}{}                                              &                                                                               & \multicolumn{1}{l}{}     &                                                                                                   &                                                                                                  & \multicolumn{1}{l}{}                                                                                                     \\
\midrule
\multirow{5}{*}{3}                                                & \begin{tabular}[c]{@{}l@{}}Contacto\\ de alarma\end{tabular}                  & Cerrado                  & \multicolumn{2}{l}{\multirow{5}{*}{\begin{tabular}[c]{@{}l@{}}Sistema monitoreado en\\ estado de alarma, al\\ menos una alarma está\\ presente en el sistema.\end{tabular}}}                         & \multirow{5}{*}{\begin{tabular}[c]{@{}c@{}}Led:\\ rojo\\ Estado RF:\\ ok\\ Estado:\\ alarma\end{tabular}}                \\
                                                                  & \multirow{4}{*}{\begin{tabular}[c]{@{}l@{}}Contacto \\ de falla\end{tabular}} & \multirow{4}{*}{Abierto} & \multicolumn{2}{l}{}                                                                                                                                                                                 &                                                                                                                          \\
                                                                  &                                                                               &                          & \multicolumn{2}{l}{}                                                                                                                                                                                 &                                                                                                                          \\
                                                                  &                                                                               &                          & \multicolumn{2}{l}{}                                                                                                                                                                                 &                                                                                                                          \\
                                                                  &                                                                               &                          & \multicolumn{2}{l}{}                                                                                                                                                                                 &                                                                                                                          \\
\multicolumn{1}{l}{}                                              &                                                                               & \multicolumn{1}{l}{}     &                                                                                                   &                                                                                                  & \multicolumn{1}{l}{}                                                                                                     \\
\midrule
\multirow{5}{*}{4}                                                & \begin{tabular}[c]{@{}l@{}}Contacto\\ de alarma\end{tabular}                  & Abierto                  & \multicolumn{2}{l}{\multirow{5}{*}{\begin{tabular}[c]{@{}l@{}}Sistema monitoreado en\\ estado de falla, al menos\\ una falla está presente\\ en el sistema.\end{tabular}}}                           & \multirow{5}{*}{\begin{tabular}[c]{@{}c@{}}Led:\\ amarillo\\ Estado RF:\\ ok\\ Estado:\\ falla\end{tabular}}             \\
                                                                  & \multirow{4}{*}{\begin{tabular}[c]{@{}l@{}}Contacto\\ de falla\end{tabular}}  & \multirow{4}{*}{Cerrado} & \multicolumn{2}{l}{}                                                                                                                                                                                 &                                                                                                                          \\
                                                                  &                                                                               &                          & \multicolumn{2}{l}{}                                                                                                                                                                                 &                                                                                                                          \\
                                                                  &                                                                               &                          & \multicolumn{2}{l}{}                                                                                                                                                                                 &                                                                                                                          \\
                                                                  &                                                                               &                          & \multicolumn{2}{l}{}                                                                                                                                                                                 &                                                                                                                          \\
\multicolumn{1}{l}{}                                              &                                                                               & \multicolumn{1}{l}{}     &                                                                                                   &                                                                                                  & \multicolumn{1}{l}{}                                                                                                    
\\
\bottomrule
\hline                                                                        
\end{tabular}
\label{tab:tabla_3}
\end{table}

\begin{figure}[ht]
	\centering
	\includegraphics[scale=.5]{./Figures/Capitulo4/Figura_J.png}
	\caption{Diagrama CTM para ensayo de estado local}
	\label{fig:figura_j}
\end{figure}

\subsection{Sistema con un dispositivo secundario.}

Utilizando la misma metodología del ensayo número uno, se generan la figura \ref{fig:figura_k} y la tabla \ref{tab:tabla_4_1} y \ref{tab:tabla_4_2}, que describen el detalle del ensayo. Se confirma que todos los casos de prueba son ejecutados de forma satisfactoria, por lo que hasta este punto se puede concluir, que el dispositivo primario es capaz de comunicarse de forma inalámbrica con un dispositivo secundario, y además puede definir el estado correspondiente del sistema de detección, a partir de su estado propio y el estado del dispositivo secundario. 

\begin{figure}[ht]
	\centering
	\includegraphics[scale=.3]{./Figures/Capitulo4/Figura_K.png}
	\caption{Diagrama CTM para ensayo con un dispositivo secundario }
	\label{fig:figura_k}
\end{figure}


\begin{table}[h]
\centering
\caption[Casos de prueba, dispositivo secundario ]{Casos de prueba del ensayo de sistema con un dispositivo primario}
\begin{tabular}{clcllc}
\toprule
\textbf{\begin{tabular}[c]{@{}c@{}}Caso de\\ prueba\end{tabular}} & \multicolumn{1}{c}{\textbf{Parámetro}}                                            & \textbf{Valor}           & \multicolumn{2}{c}{\textbf{Comentario}}                                                                                                                                                                                                                                                 & \textbf{\begin{tabular}[c]{@{}c@{}}Resultado\\ esperado\end{tabular}}                                        \\
\midrule
\multirow{5}{*}{1}   
                                             & \begin{tabular}[c]{@{}l@{}}D.P. Contacto\\ de alarma\end{tabular}                 & Abierto                  & \multicolumn{2}{l}{\multirow{5}{*}{\begin{tabular}[c]{@{}l@{}}Ambos sistemas en estado\\ normal, sin fallas ni\\ alarmas presentes.\end{tabular}}}                                                                                                                                       & \multirow{5}{*}{\begin{tabular}[c]{@{}c@{}}Led:\\ verde\\ Estado RF:\\ ok\\ Estado:\\ normal\end{tabular}}   \\
                                                                  & \begin{tabular}[c]{@{}l@{}}D.P. Contacto\\ de falla\end{tabular}                  & Abierto                  & \multicolumn{2}{l}{}                                                                                                                                                                                                                                                                    &                                                                                                              \\
                                                                  & \begin{tabular}[c]{@{}l@{}}D.S. Contacto\\ de alarma\end{tabular}                 & Abierto                  & \multicolumn{2}{l}{}                                                                                                                                                                                                                                                                    &                                                                                                              \\
                                                                  & \multirow{2}{*}{\begin{tabular}[c]{@{}l@{}}D.S. Contacto\\ de falla\end{tabular}} & \multirow{2}{*}{Abierto} & \multicolumn{2}{l}{}                                                                                                                                                                                                                                                                    &                                                                                                              \\
                                                                  &                                                                                   &                          & \multicolumn{2}{l}{}                                                                                                                                                                                                                                                                    &                                                                                                              \\
\multicolumn{1}{l}{}                                              &                                                                                   & \multicolumn{1}{l}{}     &                                                                                                                                            &                                                                                                                                            & \multicolumn{1}{l}{}                                                                                         \\
\midrule
\multirow{7}{*}{2}                                                & \begin{tabular}[c]{@{}l@{}}D.P. Contacto\\ de alarma\end{tabular}                 & Cerrado                  & \multicolumn{2}{l}{\multirow{7}{*}{\begin{tabular}[c]{@{}l@{}}Dispositivo primario: \\ sistema en estado de \\ alarma, al menos una\\ alarma presente en \\ el sistema. \\ \\ Dispositivo secundario:\\ sistema en estado \\ normal, sin fallas ni \\ alarmas presentes.\end{tabular}}} & \multirow{7}{*}{\begin{tabular}[c]{@{}c@{}}Led:\\ rojo\\ Estado RF:\\ ok\\ Estado:\\ alarma\end{tabular}}    \\
                                                                  & \begin{tabular}[c]{@{}l@{}}D.P. Contacto\\ de falla\end{tabular}                  & Abierto                  & \multicolumn{2}{l}{}                                                                                                                                                                                                                                                                    &                                                                                                              \\
                                                                  & \begin{tabular}[c]{@{}l@{}}D.S. Contacto\\ de alarma\end{tabular}                 & Abierto                  & \multicolumn{2}{l}{}                                                                                                                                                                                                                                                                    &                                                                                                              \\
                                                                  & \multirow{4}{*}{\begin{tabular}[c]{@{}l@{}}D.S. Contacto\\ de falla\end{tabular}} & \multirow{4}{*}{Abierto} & \multicolumn{2}{l}{}                                                                                                                                                                                                                                                                    &                                                                                                              \\
                                                                  &                                                                                   &                          & \multicolumn{2}{l}{}                                                                                                                                                                                                                                                                    &                                                                                                              \\
                                                                  &                                                                                   &                          & \multicolumn{2}{l}{}                                                                                                                                                                                                                                                                    &                                                                                                              \\
                                                                  &                                                                                   &                          & \multicolumn{2}{l}{}                                                                                                                                                                                                                                                                    &                                                                                                              \\
\multicolumn{1}{l}{}                                              &                                                                                   & \multicolumn{1}{l}{}     &                                                                                                                                            &                                                                                                                                            & \multicolumn{1}{l}{}                                                                                         \\
\midrule
\multirow{7}{*}{3}                                                & \begin{tabular}[c]{@{}l@{}}D.P. Contacto\\ de alarma\end{tabular}                 & Abierto                  & \multicolumn{2}{l}{\multirow{7}{*}{\begin{tabular}[c]{@{}l@{}}Dispositivo primario:\\ sistema en estado de\\ falla, al menos una\\ falla presente en\\ el sistema. \\ \\ Dispositivo secundario:\\ sistema en estado\\ normal, sin fallas\\ ni alarmas presentes.\end{tabular}}}        & \multirow{7}{*}{\begin{tabular}[c]{@{}c@{}}Led:\\ amarillo\\ Estado RF:\\ ok\\ Estado:\\ falla\end{tabular}} \\
                                                                  & \begin{tabular}[c]{@{}l@{}}D.P. Contacto\\ de falla\end{tabular}                  & Cerrado                  & \multicolumn{2}{l}{}                                                                                                                                                                                                                                                                    &                                                                                                              \\
                                                                  & \begin{tabular}[c]{@{}l@{}}D.S. Contacto\\ de alarma\end{tabular}                 & Abierto                  & \multicolumn{2}{l}{}                                                                                                                                                                                                                                                                    &                                                                                                              \\
                                                                  & \multirow{4}{*}{\begin{tabular}[c]{@{}l@{}}D.S. Contacto\\ de falla\end{tabular}} & \multirow{4}{*}{Abierto} & \multicolumn{2}{l}{}                                                                                                                                                                                                                                                                    &                                                                                                              \\
                                                                  &                                                                                   &                          & \multicolumn{2}{l}{}                                                                                                                                                                                                                                                                    &                                                                                                              \\
                                                                  &                                                                                   &                          & \multicolumn{2}{l}{}                                                                                                                                                                                                                                                                    &                                                                                                              \\
                                                                  &                                                                                   &                          & \multicolumn{2}{l}{}                                                                                                                                                                                                                                                                    &                                                                                                              \\
\multicolumn{1}{l}{}                                              &                                                                                   & \multicolumn{1}{l}{}     &                                                                                                                                            &                                                                                                                                            & \multicolumn{1}{l}{}                                                                                         \\
\midrule
\multirow{7}{*}{4}                                                & \begin{tabular}[c]{@{}l@{}}D.P. Contacto\\ de alarma\end{tabular}                 & Abierto                  & \multicolumn{2}{l}{\multirow{7}{*}{\begin{tabular}[c]{@{}l@{}}Dispositivo primario:\\ sistema en estado \\ normal, sin fallas ni\\ alarmas presentes. \\ \\ Dispositivo secundario:\\ sistema en estado de\\ alarma, al menos una\\ alarmapresente en el\\ sistema.\end{tabular}}}      & \multirow{7}{*}{\begin{tabular}[c]{@{}c@{}}Led:\\ rojo\\ Estado RF:\\ ok\\ Estado:\\ alarma\end{tabular}}    \\
                                                                  & \begin{tabular}[c]{@{}l@{}}D.P. Contacto\\ de falla\end{tabular}                  & Abierto                  & \multicolumn{2}{l}{}                                                                                                                                                                                                                                                                    &                                                                                                              \\
                                                                  & \begin{tabular}[c]{@{}l@{}}D.S. Contacto\\ de alarma\end{tabular}                 & Cerrado                  & \multicolumn{2}{l}{}                                                                                                                                                                                                                                                                    &                                                                                                              \\
                                                                  & \multirow{4}{*}{\begin{tabular}[c]{@{}l@{}}D.S. Contacto\\ de falla\end{tabular}} & \multirow{4}{*}{Abierto} & \multicolumn{2}{l}{}                                                                                                                                                                                                                                                                    &                                                                                                              \\
                                                                  &                                                                                   &                          & \multicolumn{2}{l}{}                                                                                                                                                                                                                                                                    &                                                                                                              \\
                                                                  &                                                                                   &                          & \multicolumn{2}{l}{}                                                                                                                                                                                                                                                                    &                                                                                                              \\
                                                                  &                                                                                   &                          & \multicolumn{2}{l}{}                                                                                                                                                                                                                                                                    &                                                                                                              \\
\multicolumn{1}{l}{}                                              &                                                                                   & \multicolumn{1}{l}{}     &                                                                                                                                            &                                                                                                                                            & \multicolumn{1}{l}{}                                                                                        
\\
\bottomrule
\hline                                                                        
\end{tabular}
\label{tab:tabla_4_1}
\end{table}


\begin{table}[h]
\centering
\caption[Continuación de la tabla \ref{tab:tabla_4_1}]{Casos de prueba del ensayo de sistema con un dispositivo primario}
\begin{tabular}{clcllc}
\toprule
\textbf{\begin{tabular}[c]{@{}c@{}}Caso de\\ prueba\end{tabular}} & \multicolumn{1}{c}{\textbf{Parámetro}}                                            & \textbf{Valor}           & \multicolumn{2}{c}{\textbf{Comentario}}                                                                                                                                                                                                                                                                         & \textbf{\begin{tabular}[c]{@{}c@{}}Resultado\\ esperado\end{tabular}}                                                    \\
\midrule
\multirow{8}{*}{5}                                                & \begin{tabular}[c]{@{}l@{}}D.P. Contacto\\ de alarma\end{tabular}                 & Abierto                  & \multicolumn{2}{l}{\multirow{8}{*}{\begin{tabular}[c]{@{}l@{}}Dispositivo primario:\\ sistema en estado\\ normal, sin fallas ni\\ alarmas presentes. \\ \\ Dispositivo secundario:\\ sistema en estado de\\ alarma y falla, al menos\\ una alarma y una falla\\ están presentes en el\\ sistema.\end{tabular}}} & \multirow{8}{*}{\begin{tabular}[c]{@{}c@{}}Led:\\ rojo-amarillo\\ Estado RF:\\ ok\\ Estado:\\ alarma-falla\end{tabular}} \\
                                                                  & \begin{tabular}[c]{@{}l@{}}D.P. Contacto\\ de falla\end{tabular}                  & Abierto                  & \multicolumn{2}{l}{}                                                                                                                                                                                                                                                                                            &                                                                                                                          \\
                                                                  & \begin{tabular}[c]{@{}l@{}}D.S. Contacto\\ de alarma\end{tabular}                 & Cerrado                  & \multicolumn{2}{l}{}                                                                                                                                                                                                                                                                                            &                                                                                                                          \\
                                                                  & \multirow{5}{*}{\begin{tabular}[c]{@{}l@{}}D.S. Contacto\\ de falla\end{tabular}} & \multirow{5}{*}{Cerrado} & \multicolumn{2}{l}{}                                                                                                                                                                                                                                                                                            &                                                                                                                          \\
                                                                  &                                                                                   &                          & \multicolumn{2}{l}{}                                                                                                                                                                                                                                                                                            &                                                                                                                          \\
                                                                  &                                                                                   &                          & \multicolumn{2}{l}{}                                                                                                                                                                                                                                                                                            &                                                                                                                          \\
                                                                  &                                                                                   &                          & \multicolumn{2}{l}{}                                                                                                                                                                                                                                                                                            &                                                                                                                          \\
                                                                  &                                                                                   &                          & \multicolumn{2}{l}{}                                                                                                                                                                                                                                                                                            &                                                                                                                          \\
\multicolumn{1}{l}{}                                              &                                                                                   & \multicolumn{1}{l}{}     &                                                                                                                                                        &                                                                                                                                                        & \multicolumn{1}{l}{}                                                                                                     \\
\midrule
\multirow{7}{*}{6}                                                & \begin{tabular}[c]{@{}l@{}}D.P. Contacto\\ de alarma\end{tabular}                 & Cerrado                  & \multicolumn{2}{l}{\multirow{7}{*}{\begin{tabular}[c]{@{}l@{}}Dispositivo primario:\\ sistema en estado\\ de alarma, al menos\\ una alarma presente\\ en el sistema.\\ \\ Dispositivo secundario:\\ sistema en estado de\\ falla, al menos una falla\\ presente en el sistema.\end{tabular}}}                   & \multirow{7}{*}{\begin{tabular}[c]{@{}c@{}}Led:\\ rojo-amarillo\\ Estado RF:\\ ok\\ Estado:\\ alarma-falla\end{tabular}} \\
                                                                  & \begin{tabular}[c]{@{}l@{}}D.P. Contacto\\ de falla\end{tabular}                  & Abierto                  & \multicolumn{2}{l}{}                                                                                                                                                                                                                                                                                            &                                                                                                                          \\
                                                                  & \begin{tabular}[c]{@{}l@{}}D.S. Contacto\\ de alarma\end{tabular}                 & Abierto                  & \multicolumn{2}{l}{}                                                                                                                                                                                                                                                                                            &                                                                                                                          \\
                                                                  & \multirow{4}{*}{\begin{tabular}[c]{@{}l@{}}D.S. Contacto\\ de falla\end{tabular}} & \multirow{4}{*}{Cerrado} & \multicolumn{2}{l}{}                                                                                                                                                                                                                                                                                            &                                                                                                                          \\
                                                                  &                                                                                   &                          & \multicolumn{2}{l}{}                                                                                                                                                                                                                                                                                            &                                                                                                                          \\
                                                                  &                                                                                   &                          & \multicolumn{2}{l}{}                                                                                                                                                                                                                                                                                            &                                                                                                                          \\
                                                                  &                                                                                   &                          & \multicolumn{2}{l}{}                                                                                                                                                                                                                                                                                            &                                                                                                                          \\
\multicolumn{1}{l}{}                                              &                                                                                   & \multicolumn{1}{l}{}     &                                                                                                                                                        &                                                                                                                                                        & \multicolumn{1}{l}{}                                                                                                     \\
\midrule
\multirow{8}{*}{7}                                                & \begin{tabular}[c]{@{}l@{}}D.P. Contacto\\ de alarma\end{tabular}                 & Cerrado                  & \multicolumn{2}{l}{\multirow{8}{*}{\begin{tabular}[c]{@{}l@{}}Dispositivo primario: \\ sistema en estado \\ de alarma, al menos\\ una alarma presente\\ en el sistema.\\ \\ Dispositivo secundario:\\ sistema en estado de\\ alarma, al menos una\\ alarmapresente en el\\ sistema.\end{tabular}}}              & \multirow{8}{*}{\begin{tabular}[c]{@{}c@{}}Led:\\ rojo\\ Estado RF:\\ ok\\ Estado:\\ alarma\end{tabular}}                \\
                                                                  & \begin{tabular}[c]{@{}l@{}}D.P. Contacto\\ de falla\end{tabular}                  & Abierto                  & \multicolumn{2}{l}{}                                                                                                                                                                                                                                                                                            &                                                                                                                          \\
                                                                  & \begin{tabular}[c]{@{}l@{}}D.S. Contacto\\ de alarma\end{tabular}                 & Cerrado                  & \multicolumn{2}{l}{}                                                                                                                                                                                                                                                                                            &                                                                                                                          \\
                                                                  & \multirow{5}{*}{\begin{tabular}[c]{@{}l@{}}D.S. Contacto\\ de falla\end{tabular}} & \multirow{5}{*}{Abierto} & \multicolumn{2}{l}{}                                                                                                                                                                                                                                                                                            &                                                                                                                          \\
                                                                  &                                                                                   &                          & \multicolumn{2}{l}{}                                                                                                                                                                                                                                                                                            &                                                                                                                          \\
                                                                  &                                                                                   &                          & \multicolumn{2}{l}{}                                                                                                                                                                                                                                                                                            &                                                                                                                          \\
                                                                  &                                                                                   &                          & \multicolumn{2}{l}{}                                                                                                                                                                                                                                                                                            &                                                                                                                          \\
                                                                  &                                                                                   &                          & \multicolumn{2}{l}{}                                                                                                                                                                                                                                                                                            &                                                                                                                          \\
\multicolumn{1}{l}{}                                              &                                                                                   & \multicolumn{1}{l}{}     &                                                                                                                                                        &                                                                                                                                                        & \multicolumn{1}{l}{}                                                                                                     \\
\midrule
\multirow{6}{*}{8}                                                & \begin{tabular}[c]{@{}l@{}}D.P. Contacto\\ de alarma\end{tabular}                 & Abierto                  & \multicolumn{2}{l}{\multirow{6}{*}{\begin{tabular}[c]{@{}l@{}}Dispositivo primario:\\ sistema en estado\\ de falla, al menos\\ una falla presente\\ en el sistema.\\ \\ Dispositivo secundario:\\ sistema en estado de\\
alarma.\end{tabular}}}                 & \multirow{6}{*}{\begin{tabular}[c]{@{}c@{}}Led:\\ rojo-amarillo\\ Estado RF:\\ ok\\ Estado:\\ alarma-falla\end{tabular}} \\
                                                                  & \begin{tabular}[c]{@{}l@{}}D.P. Contacto\\ de falla\end{tabular}                  & Cerrado                  & \multicolumn{2}{l}{}                                                                                                                                                                                                                                                                                            &                                                                                                                          \\
                                                                  & \begin{tabular}[c]{@{}l@{}}D.S. Contacto\\ de alarma\end{tabular}                 & Cerrado                  & \multicolumn{2}{l}{}                                                                                                                                                                                                                                                                                            &                                                                                                                          \\
                                                                  & \multirow{3}{*}{\begin{tabular}[c]{@{}l@{}}D.S. Contacto\\ de falla\end{tabular}} & \multirow{3}{*}{Abierto} & \multicolumn{2}{l}{}                                                                                                                                                                                                                                                                                            &                                                                                                                          \\
                                                                  &                                                                                   &                          & \multicolumn{2}{l}{}                                                                                                                                                                                                                                                                                            &                                                                                                                          \\
                                                                  &                                                                                   &                          & \multicolumn{2}{l}{}                                                                                                                                                                                                                                                                                            &                                                                                                                          \\
\multicolumn{1}{l}{}                                              &                                                                                   & \multicolumn{1}{l}{}     &                                                                                                                                                        &                                                                                                                                                        & \multicolumn{1}{l}{}                                                                                                    
\\
\bottomrule
\hline                                                                        
\end{tabular}
\label{tab:tabla_4_2}
\end{table}

\subsection{Sistema con dos dispositivos secundarios.}

Se mantiene la estrategia para el diseño de casos de prueba y se obtiene como resultado la figura \ref{fig:figura_l}, pero a diferencia del ensayo anterior, para este ensayo se incluye un nodo que permanecerá en estado de alarma durante todos los casos de prueba. Los resultados en esta ocasión son negativos, ya que ninguno de los casos de prueba es cumplido.

\begin{figure}[ht]
	\centering
	\includegraphics[scale=.3]{./Figures/Capitulo4/Figura_L.png}
	\caption{Diagrama CTM para sistema con dos dispositivos secundarios}
	\label{fig:figura_l}
\end{figure}

Los casos de prueba fallan debido a que el sistema es incapaz de responder de forma correcta en un tiempo menor a diez segundos, analizando el algoritmo de gestión de comunicación inalámbrica en el dispositivo primario y compararlo con el algoritmo de transmisión de paquetes, se descubrió que el dispositivo secundario está configurado para la transmisión de información cada 15 ms y el dispositivo primario disponía de 10 ms como ventana de tiempo para hacer el análisis del sistema,  lo que genera un comportamiento inestable en el sistema, al no poder asegurar que el sistema reciba la información de todos los nodos que conforman la red en una ventana tan corta de tiempo.
Un problema adicional que se presentó por la lectura inadecuada de los mensajes recibidos por el transceptor, fue la acumulación de mensajes, lo que ocasiona retrasos en la respuesta del sistema, que se agravaba a medida que el sistema se mantuvo en funcionamiento.

 Al identificar la causa del problema, se planteó una modificación de los tiempos configurados para la comunicación inalámbrica en ambos dispositivos, de manera que se evite la desincronización del sistema. Los nuevos parámetros establecidos para la comunicación, son 200 ms para el dispositivo primario y 30 ms para el dispositivo secundario, esta nueva configuración permitió cumplir correctamente los casos de prueba. 
      
Un punto de mejora que resulta de la realización de este ensayo, es la posibilidad de incluir una interfaz que permita facilitar la inclusión de dispositivos al sistema, ya que la metodología actual de trabajo, hace imposible al usuario incluir un nodo sin conocer el funcionamiento del código y su estructura.



\subsection{Sistema con tres dispositivos secundarios.}

Basados en la información recopilada de los ensayos previos, se propone elaborar un nuevo ensayo con un nodo adicional, pero esta vez con la intención de validar si la configuración de tiempo seleccionada es adecuada. El ensayo incluye de forma similar al ensayo número tres un nodo con un estado de sistema estático, con la diferencia de que en este caso el nuevo nodo permanecerá en estado normal, durante todos los casos de prueba, con el propósito de poder validar la velocidad de respuesta del sistema. El ensayo logró confirmar que el sistema funciona correctamente con los parámetros fijados en el ensayo con dos dispositivos secundarios.



\section{Prueba de sistema}


Los resultados obtenidos hasta este punto manifestaron errores y segmentos de código que debían ser atendidos, por lo que la finalidad de este ensayo es evaluar el impacto de los cambios realizados. Los casos de prueba de este sistema serán los descritos en las tablas \ref{tab:tabla_4_1} y \ref{tab:tabla_4_2}, correspondientes al ensayo con un dispositivo secundario, con la diferencia de que en esta ocasión, se incluyen las siguientes características al sistema: 

\subsubsection{Interfaz web}
Interfaz web para la visualización del estado actual del sistema, desarrollado en la plataforma Node-RED, con la limitación de que debemos encontrarnos conectados a la misma red que el dispositivo primario. Un ejemplo de la misma puede observarse en la la figura \ref{fig:figura_m}.


\begin{figure}[ht]
	\centering
	\includegraphics[scale=.55]{./Figures/Capitulo4/Figura_M.png}
	\caption{Interfaz web desarrollada para monitoreo del sistema utilizando la plataforma Node-RED}
	\label{fig:figura_m}
\end{figure}

\subsubsection{Aplicación Android}
Aplicación móvil para dispositivos Android, conectada a la plataforma de Firebase para el monitoreo del sistema desde cualquier ubicación con conexión a internet, en la \ref{fig:figura_n} se muestra una captura de pantalla de la interfaz.

\begin{figure}[ht]
	\centering
	\includegraphics[scale=.35]{./Figures/Capitulo4/Figura_N.png}
	\caption{Aplicación móvil desarrollado con la plataforma MIT App Inventor, para visualización de datos cargados en el servidor web de Firebase.}
	\label{fig:figura_n}
\end{figure}

\subsubsection{Histórico de eventos como base de datos}
Registro de eventos en bases de datos relacionales, esta funcionalidad permite al sistema registrar de forma ordenada los eventos y la posibilidad de analizar el histórico de eventos del sistema con el uso del lenguaje SQL. La figura \ref{fig:figura_p} corresponde a un ejemplo de la base de datos generada.

\begin{figure}[ht]
	\centering
	\includegraphics[scale=.45]{./Figures/Capitulo4/Figura_P.png}
	\caption{Ejemplo de histórico de eventos registrados en la base de dato.}
	\label{fig:figura_p}
\end{figure}

\subsubsection{Estado de red de dispositivos en base de datos}
Registro de nodos conectados al sistema en bases de datos relacionales, similar a la base de datos de registro de eventos, se incluye una base de datos con el registro de los nodos conectados, su estado actual y si se encuentran comunicándose correctamente con el dispositivo primario o no, se puede ver un ejemplo de esta base de datos en la figura \ref{fig:figura_o}.


\begin{figure}[ht]
	\centering
	\includegraphics[scale=.45]{./Figures/Capitulo4/Figura_O.png}
	\caption{Ejemplo de la base de datos generada de una red de dispositivos inalámbricos .}
	\label{fig:figura_o}
\end{figure}


\subsubsection{Estado de red de dispositivos en base de datos}
textit{Logging} gestionado por la biblioteca textit{syslog}, se hace uso del recurso del archivo textit{syslog} para facilitar el análisis en caso de fallas, registro de tareas de mantenimiento del sistema y uso como herramienta de desarrollo para futuros proyectos. Una vista del textit{syslog} del sistema en funcionamiento se puede apreciar en la figura \ref{fig:figura_q}.


\begin{figure}[ht]
	\centering
	\includegraphics[scale=.65]{./Figures/Capitulo4/Figura_Q.png}
	\caption{Vista del archivo textit{syslog} y los mensajes registrados por sistema de monitoreo.}
	\label{fig:figura_q}
\end{figure}

 
% Chapter Template

\chapter{Conclusiones} % Main chapter title

\label{Chapter5} % Change X to a consecutive number; for referencing this chapter elsewhere, use \ref{ChapterX}


%----------------------------------------------------------------------------------------

%----------------------------------------------------------------------------------------
%	SECTION 1
%----------------------------------------------------------------------------------------

\section{Conclusiones generales }

El trabajo finaliza con el desarrollo de un sistema de monitoreo remoto, compatible con sistemas de detección que funcionen con contactos secos programables, además cuenta con la capacidad de poder establecer comunicación inalámbrica con diferentes dispositivos, en una estructura de red de nodos repetidores. Cada nodo ejecuta un protocolo de recuperación ante pérdidas de comunicación, que les permite restaurarse de forma automática ante diferentes eventos de falla, evitando en la gran mayoría de ocasiones el traslado de personal técnico al lugar de la instalación. 

Este sistema fue puesto a prueba en obra para validar su correcto funcionamiento, se utilizaron dispositivos con acceso web y se comprobó que el usuario pudiese visualizar el estado actual  de su sistema de seguridad de forma remota. 

Se resaltan las prácticas que fueron clave para cada etapa del proyecto:

\begin{itemize}
\item Planificación: aunque la planificación no se ejecutó según lo planteado originalmente, se considera una herramienta indispensable, ya que permite monitorear el progreso de las diferentes actividades que  integran el proyecto, así como identificar  retrasos  y/o  tareas que requieren mayor dedicación, lo cual permite  tomar las acciones necesarias en el tiempo adecuado para cumplir con los objetivos planteados.
\item Diseño: utilizar un patrón de diseño, combinado con una estrategia de desarrollo de software modularizado, permitió generar una estructura de código enfocado principalmente en escalabilidad. Este diseño evolucionó en una metodología de desarrollo, fundamentado en la elaboración de pequeños bloques funcionales probados con elementos de testing. Lo cual facilitó la inclusión de los conocimientos adquiridos a lo largo de la carrera de forma organizada.  
\item Software: a partir de las experiencias previas con sistemas operativos, fue posible realizar dos dispositivos funcionales, basados en sistemas operativos diferentes. El dispositivo primario aprovecha la gestión de un sistema operativo de propósito general para establecer una interfaz web bidireccional, asimismo el dispositivo secundario utiliza un sistema operativo cooperativo, que brinda compatibilidad con diferentes microcontroladores y a la vez escalabilidad al proyecto.   
\item Hardware: un aporte invaluable para el trabajo fueron los conocimientos adquiridos en el curso de diseño de circuitos impresos, donde se realizaron cambios necesarios al diseño original, mejorando la robustez, manufactura, facilitando el futuro mantenimiento y generando documentación adecuada para el proyecto.
\end{itemize}


%----------------------------------------------------------------------------------------
%	SECTION 2
%----------------------------------------------------------------------------------------
\section{Próximos pasos}

El sistema actual cumple con los requerimientos planteados inicialmente. Sin embargo, durante su desarrollo surgieron puntos de mejora y trabajos a desarrollar, entre los cuales se resaltan los siguientes:

\begin{itemize}
\item Aprovechar la conectividad WiFi del dispositivo secundario, para establecer una vía alterna para la carga del estado del sistema en el servidor web, en caso de fallas en el dispositivo primario.
\item Desarrollo de una interfaz web personalizada, que complemente los avances realizados en la plataforma Node-RED. 
\item Implementación de un servidor que regule el acceso de los usuarios, que además permita visualizar la información desde cualquier punto con conectividad a internet.
\item Elaboración de aplicaciones Android e IOS con la capacidad de generar notificaciones, ante el arribo de nuevos eventos. 
\end{itemize}

El sistema actual se basa en el monitoreo de contactos secos programables, con el objetivo de mantener compatibilidad con diferentes marcas comerciales. A la vez existen segmentos del mercado con requerimientos más exigentes. Algunos posibles proyectos son:


\begin{itemize}
\item Hacer uso del puerto de conexión de impresoras de algunas marcas comerciales de centrales de alarma de incendio, para obtener un mayor detalle de los eventos que ocurren en la instalación.
\item Diseñar una interfaz gráfica que se adecue a cada obra, facilitando al usuario la documentación de la instalación, tal como información de los equipos instalados, instrucciones ante diferentes eventos, planos con el estado de real de los dispositivos, datos históricos, próximos mantenimientos, etc.
\end{itemize}

Un aporte significativo que se espera poder alcanzar con esta serie de proyectos, consiste en establecer un sistema con la robustez y confiabilidad suficiente, para instaurar un vínculo con personal capacitado. Lo cual puede favorecer a generar una respuesta temprana ante situaciones de incendio, disminuyendo los daños ocasionados y salvaguardando de manera eficaz la vida de las personas en situaciones de emergencia. 
 

%----------------------------------------------------------------------------------------
%	CONTENIDO DE LA MEMORIA  - APÉNDICES
%----------------------------------------------------------------------------------------

\appendix % indicativo para indicarle a LaTeX los siguientes "capítulos" son apéndices

% Incluir los apéndices de la memoria como archivos separadas desde la carpeta Appendices
% Descomentar las líneas a medida que se escriben los apéndices

%\include{Appendices/AppendixA}
%\include{Appendices/AppendixB}
%\include{Appendices/AppendixC}

%----------------------------------------------------------------------------------------
%	BIBLIOGRAPHY
%----------------------------------------------------------------------------------------

\Urlmuskip=0mu plus 1mu\relax
\raggedright
\printbibliography[heading=bibintoc]

%----------------------------------------------------------------------------------------

\end{document}  
